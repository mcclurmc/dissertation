\chapter{Background}
\label{background}

In this section we will provide a brief tour of the OCaml programming
language, focusing espectially on its powerful module system. The
chapter follows with an in-depth discussion of xUnit's Test Double
patterns and their implementation in OCaml and Java. The chapter
concludes with a summary of the current state-of-the-art testing tools
for functional languages such as OCaml and Haskell.

\section{A brief tour of OCaml and its module system}
\label{ocaml}

%% XXX try this:
%% https://tex.stackexchange.com/questions/106844/adding-words-to-lstlisting-for-python-langaue
%% in order to get multiple levels of keywords
\lstset{
  language=ML,
  morekeywords={function,module},
}

%% Want to talk quickly about code (functions, types, type inference),
%% modules, functors, first class modules. Mention that OCaml has a class
%% language, but that idiomatic OCaml typically uses functions and
%% modules unless a particular algorithm or solution to a problem
%% warrants an object-oriented implemtation.

OCaml is a strongly, statically typed functional programming language
based on the ML family of languages. It may be interpreted in a
read-eval-print loop called the OCaml ``toplevel,'' or it may be
compiled to an OCaml specific bytecode languate or to a native
executable. \cite{ocaml:spec} It has a very good generational garbage
collector tuned for sequential programs. \cite{ocaml:gc_tutorial}

What follows is a \textit{very} brief tour of the OCaml programming
language. This should hopefully be enough to allow the reader to
follow along with the coding examples later in this chapter and the
next. For more assistance with OCaml, the reader is directed to the
OCaml Langauge Specification \cite{ocaml:spec}, and Real World OCaml
\cite{rwo}\footnote{Real World OCaml is available for free online:
  \url{https://realworldocaml.org/v1/en/html/index.html}}

\subsubsection{OCaml basics: values, functions, and types}

In OCaml, values are assigned to variables using the \code{let}
keyword.

\begin{lstlisting}
let num = 42
let str = "hello world"
let list = [1;2;3;4]
let tuple = (1,2)
\end{lstlisting}

Variables in OCaml are immutable, unless they are specially defined as
references, or mutable fields of a record type.

\begin{lstlisting}
let num = ref 42
Printf.printf "num was %d\n" !num
num := 24
Printf.printf "num is now %d\n" !num
\end{lstlisting}

Functions are created with the keywords \code{fun} or \code{function},
and assigned names using \code{let}, just like other values. Functions
defined using the \code{function} keyword can only take one arguement,
but that argument can be pattern-matched.

\begin{lstlisting}
let f = fun a b -> a + b
let g = function
        | [] -> true
        | _  -> false
\end{lstlisting}

Functions can also be created using a shorthand \code{let} syntax. The
function \code{f} below is equivalent to the function \code{f} above.

\begin{lstlisting}
let f a b = a + b
\end{lstlisting}

OCaml is a strongly, statically typed language. The compiler will
infer the type of values, and will return an error if a type
constraint is violated. For instance, the following function \code{f}
has type \code{int -> int -> int}, meaning that it takes two values of
type \code{int} and returns a value of type \code{int}. If a
\code{string} or \code{float} are passed to this function, a
compilation error is thrown.

\begin{lstlisting}
let f a b = a + b     (* has type int -> int -> int *)
f 40 2                (* this type-checks *)
f "4" "2"             (* this fails to type-check *)
f 41.8 0.2            (* this also fails to type-check *)
\end{lstlisting}

Because functions are first-class values just like \code{int}s and
\code{string}s, we can pass them to other functions. For instance, the
function \code{List.map} takes a function \code{f} and a list
\code{ls}, and returns a new list containing the result of applying
\code{f} to each element of \code{ls}.

\begin{lstlisting}
let f x = x + 1
let ls = [1;2;3;4]
List.map f ls         (* returns the value [2;3;4;5] *)
\end{lstlisting}

OCaml allows the user to define complex types.

\begin{lstlisting}
(* Variant types *)
type colors =
  | Red of int
  | Green of int
  | Blue of int

(* Polymorphic variants *)
type json = ...

(* Record types *)
type book = {
  author : string;
  title  : string;
  mutable inventory : int;
}
\end{lstlisting}

XXX type variables

XXX pattern matching

\begin{lstlisting}
match [1;2;3] with
| [] -> "empty"
| 1::ls -> ...
\end{lstlisting}

\subsubsection{The OCaml module system}

A distinguishing feature of OCaml is its powerful module system. A
module in OCaml is the basic compilation unit. Each compiled file is
assigned its own module. For instance, the definitions inside a file
called \code{foo.ml} would be placed into a module \code{Foo}.

%% XXX not sure if we want these to come from separate files, or if we
%% want to put a border around them, or...
\lstinputlisting[caption=foo.ml,frame=single]{examples/foo.ml}

From outside of this module, function \code{f} would be referenced as
\code{Foo.f}. Module \code{Foo} has the following type:

\begin{lstlisting}
type t = int
val f : 'a -> 'a
\end{lstlisting}

The types of modules can be restricted useing \code{mli} files. When
an \code{mli} file exists with the same basename as an \code{ml} file,
the module type of the module created for that \code{ml} file is
restricted to that of the signature specified in the \code{mli}
file. For instance, if \code{foo.mli} is:

\lstinputlisting[caption=foo.mli]{examples/foo.mli}

then the resulting type of module \code{Foo} would be:

\begin{lstlisting}
type t
val f : t -> t
\end{lstlisting}

Notice that \code{type t} is now abstract (meaning that we can't know
its implementation outside of module \code{Foo}, and function \code{f}
is now restricted to have type \code{t -> t}, instead of the more
general \code{'a -> 'a}.

XXX smaller modules

OCaml modules can also be defined independently of compilation
units.

\begin{lstlisting}
module Bar =
struct
  let g a = a + 1
  let h ls = List.map g ls
end
\end{lstlisting}

We can also create module type signatures, which can restrict the
resulting type of a module in the same way as an \code{mli} file.

\begin{lstlisting}
module type BAR =
sig
  val h : int list -> int list
end
\end{lstlisting}

XXX Show how to restrict \code{module Bar2 = (Bar : BAR)}

XXX functors

XXX first class modules

\subsubsection{The rest of OCaml}

Briefly mention OCaml's object system and class language, and point
the reader to books and web resources.

\section{Test Double patterns in detail}
\label{testdoubles}

\begin{figure}
  \centering
  \includegraphics[scale=1.0]{img/XXX.png}
  \caption[Taxonomy of Test Double patterns]{Taxonomy of Test Double patterns\footnotemark}
  \label{fig:taxonomy}
\end{figure}
\footnotetext{Adapted from Meszaros \cite{meszaros:xunit}}

As figure \ref{fig:taxonomy} shows\dots

\subsection{Dummy Object pattern}
\label{testdoubles:dummy}

\subsection{Fake Object pattern}
\label{testdoubles:fake}

\subsection{Test Stub pattern}
\label{testdoubles:stub}

\subsection{Test Spy Object pattern}
\label{testdoubles:spy}

\subsection{Mock Object pattern}
\label{testdoubles:mocks}

\section{Tools for software testing in functional langauges}
\label{testtools}

\begin{enumerate}
\item Haskell quickcheck \cite{claessen:quickcheck}
\item xUnit implementations: HUnit \cite{www:hunit}, OUnit \cite{www:ounit}
\item Kaputt for OCaml \cite{www:kaputt}
\item Haskell benchmarking lib? \cite{www:criterion}
\end{enumerate}
