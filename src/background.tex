\chapter{Background}
\label{background}

In this section we will provide a brief tour of the OCaml programming
language, focusing espectially on its powerful module system. The
chapter follows with an in-depth discussion of xUnit's Test Double
patterns and their implementation in OCaml and Java. The chapter
concludes with a summary of the current state-of-the-art testing tools
for functional languages such as OCaml and Haskell.

\section{A brief tour of OCaml and its module system}
\label{ocaml}

%% XXX try this:
%% https://tex.stackexchange.com/questions/106844/adding-words-to-lstlisting-for-python-langaue
%% in order to get multiple levels of keywords
\lstset{
  language=ML,
  morekeywords={function,module,match,try},
}

%% Want to talk quickly about code (functions, types, type inference),
%% modules, functors, first class modules. Mention that OCaml has a class
%% language, but that idiomatic OCaml typically uses functions and
%% modules unless a particular algorithm or solution to a problem
%% warrants an object-oriented implemtation.

OCaml is a strongly, statically typed functional programming language
based on the ML family of languages. It may be interpreted in a
read-eval-print loop called the OCaml ``toplevel,'' or it may be
compiled to an OCaml specific bytecode languate or to a native
executable. \cite{ocaml:spec} It has a very good generational garbage
collector tuned for sequential programs. \cite{ocaml:gc_tutorial}

What follows is a \textit{very} brief tour of the OCaml programming
language. This should hopefully be enough to allow the reader to
follow along with the coding examples later in this chapter and the
next. For more assistance with OCaml, the reader is directed to the
OCaml Langauge Specification \cite{ocaml:spec}, and Real World OCaml
\cite{rwo}\footnote{Real World OCaml is available for free online:
  \url{https://realworldocaml.org}}

\subsection{OCaml basics: values, functions, and types}

In OCaml, values are assigned to variables using the \code{let}
keyword.

\begin{lstlisting}
(* This is a comment *)
let num = 42
let str = "hello world"
let list = [1;2;3;4]
let tuple = (1,2)
\end{lstlisting}

Variables in OCaml are immutable, unless they are specially defined as
references, or mutable fields of a record type.

\begin{lstlisting}
let num = ref 42
Printf.printf "num was %d\n" !num
num := 24
Printf.printf "num is now %d\n" !num
\end{lstlisting}

Functions are created with the keywords \code{fun} or \code{function},
and assigned names using \code{let}, just like other values. Functions
defined using the \code{function} keyword can only take one arguement,
but that argument can be pattern-matched.

\begin{lstlisting}
let f = fun a b -> a + b
let g = function
        | [] -> true
        | _  -> false
\end{lstlisting}

Functions can also be created using a shorthand \code{let} syntax. The
function \code{f} below is equivalent to the function \code{f} above.

\begin{lstlisting}
let f a b = a + b
\end{lstlisting}

OCaml is a strongly, statically typed language. The compiler will
infer the type of values, and will return an error if a type
constraint is violated. For instance, the following function \code{f}
has type \code{int -> int -> int}, meaning that it takes two values of
type \code{int} and returns a value of type \code{int}. If a
\code{string} or \code{float} are passed to this function, a
compilation error is thrown.

\begin{lstlisting}
let f a b = a + b     (* has type int -> int -> int *)
f 40 2                (* this type-checks *)
f "4" "2"             (* this fails to type-check *)
f 41.8 0.2            (* this also fails to type-check *)
\end{lstlisting}

Because functions are first-class values just like \code{int}s and
\code{string}s, we can pass them to other functions. For instance, the
function \code{List.map} takes a function \code{f} and a list
\code{ls}, and returns a new list containing the result of applying
\code{f} to each element of \code{ls}.

\begin{lstlisting}
let f x = x + 1
let ls = [1;2;3;4]
List.map f ls         (* returns the value [2;3;4;5] *)
\end{lstlisting}

OCaml allows the user to define complex types.

\begin{lstlisting}
(* Tuples *)
type position = int * int

(* Variant types *)
type colors =
  | Red of int
  | Green of int
  | Blue of int

(* Record types *)
type book = {
  author : string;
  title  : string;
  mutable inventory : int;
}
\end{lstlisting}

%% polymorphic variants

\begin{lstlisting}
(* Polymorphic variants *)
type json = XXX
\end{lstlisting}

%% type variables

Polymorphism in OCaml data types is expressed with type variables. In
the following example, \code{'a} is a type variable which can
represent any type, making \code{tree} a generic data type.

\begin{lstlisting}
type 'a tree =
  | Leaf of 'a
  | Node of 'a tree * 'a tree
\end{lstlisting}

%% pattern matching

One of the most powerful features of languages like OCaml is pattern
matching. OCaml allows one to match over the structure of defined data
types. Pattern matching can be accomplished with the
\code{match value with | patt -> expr}
expression, or in functions created with the \code{function} keyword.

\begin{lstlisting}
let rec num_leaves = function
  | Leaf _      -> 1
  | Node (l, r) ->
    (num_leaves l) + (num_leaves r)
\end{lstlisting}

%% exceptions

\begin{lstlisting}
exception Not_found
let rec find x = function
  | [] -> raise Not_found
  | (k,v)::ls -> if x = k
                 then v
                 else find x ls
\end{lstlisting}

Exceptions can also carry data:

\begin{lstlisting}
exception Error of [`Invalid_request of string | `Permissions]
raise (Error (`Invalid_request "foo"))
raise (Error `Permissions)
\end{lstlisting}

``try/catch'' blocks in OCaml are similar to those in other languages,
except OCaml pattern matches on the ``catch'' block using the
\code{with} keyword.

\begin{lstlisting}
try
  raise (Error (`Invalid_request "foo"))
with
  | Error `Permissions ->
    print_endline "Bad permissions"
  | Error (`Invalid_request req) ->
    Printf.printf "Invalid request: %s\n" req
\end{lstlisting}

\subsection{The OCaml module system}

\subsubsection{Modules, signatures, and compilation units}

A distinguishing feature of OCaml is its powerful module system. A
module in OCaml is the basic compilation unit. Each compiled file is
assigned its own module. For instance, the definitions inside a file
called \code{foo.ml} would be placed into a module \code{Foo}.

%% XXX not sure if we want these to come from separate files, or if we
%% want to put a border around them, or...
\lstinputlisting[
  caption=foo.ml,
  %frame=single,
]{code/foo.ml}

From outside of this module, function \code{f} would be referenced as
\code{Foo.f}. Module \code{Foo} has the following type:

\begin{lstlisting}
type t = int
val f : 'a -> 'a
\end{lstlisting}

The types of modules can be restricted useing \code{mli} files. When
an \code{mli} file exists with the same basename as an \code{ml} file,
the module type of the module created for that \code{ml} file is
restricted to that of the signature specified in the \code{mli}
file. For instance, if \code{foo.mli} is:

\lstinputlisting[caption=foo.mli]{code/foo.mli}

then the resulting type of module \code{Foo} would be:

\begin{lstlisting}
type t
val f : t -> t
\end{lstlisting}

Notice that \code{type t} is now abstract (meaning that we can't know
its implementation outside of module \code{Foo}, and function \code{f}
is now restricted to have type \code{t -> t}, instead of the more
general \code{'a -> 'a}.

OCaml modules can also be defined independently of compilation
units.

\begin{lstlisting}
module Bar =
struct
  type t = int
  let g a = a + 1
  let h ls = List.map g ls
end
\end{lstlisting}

We can also create module type signatures, which can restrict the
resulting type of a module in the same way as an \code{mli} file.

\begin{lstlisting}
module type BAR =
sig
  type t
  val h : int list -> int list
end
\end{lstlisting}

We can limit the type of \code{module Bar} to the signature
\code{BAR} by restricting it:

\begin{lstlisting}
module Bar2 = (Bar : BAR)
\end{lstlisting}

We can also restrict the orignal definition of \code{module Bar}:

\begin{lstlisting}
module Bar1 : BAR =
struct
  type t = int
  let g a = a + 1
  let h ls = List.map g ls
end
\end{lstlisting}

Modules \code{Bar1} and \code{Bar2} are now restricted by the
signature \code{BAR}. The value \code{g} is now hidden, and
\code{type t} is now abstract.

\subsubsection{Functors}

(XXX functors) In most languages, a module system is primarily meant
to provide a namespacing system or a way to refer to compilation units
(cf. Haskell's module system \cite{www:haskell:modules}, or Python's
module system \cite{www:python:modules}). OCaml, however, has a much
more powerful module system, including the ability to define
\textit{functors}.

A functor is a relation from modules to modules. A functor can take a
module as an arguement, and use it to specialise another module. For
instance, here is an example adapted from the OCaml language
``Introduction to OCaml'' \cite{ocaml:spec} that creates a \code{Set}
functor.

\begin{lstlisting}
module type ORDERED_TYPE =
sig
  type t
  val compare: t -> t -> int
end

module Set =
  functor (Elt: ORDERED_TYPE) ->
    struct
      type element = Elt.t
      type set = element list
      let empty = []
      let add x s = ...
    end
\end{lstlisting}

\code{Set} is a functor that takes some module with same type
signature as \code{ORDERED_TYPE}, and returns a module which is
specialised with that element type. For instance, we may create a new
\code{StringSet} module using the \code{Set} functor and the
\code{String} module from the OCaml standard library.

\begin{lstlisting}
module StringSet = Set(String)

let set = StringSet.add "foo" StringSet.empty in
StringSet.mem "foo" set  (* => true *)
\end{lstlisting}

For brevity, OCaml supports and alternative syntax for functor
definition which is more concise. For instance, we could have defined
the \code{Set} functor as:
\code{module Set(Elt : ORDERED_TYPE) = struct ... end}.
This comes in handy when one is defining a functor which takes
multiple arguements:
\code{module F(A : X)(B : Y) = struct ... end} .

Functors are useful for creating generic data types, such as the
\code{Set} type we defined above. We will soon see that they can also
be useful for dependency injection.

\subsubsection{First-class modules}

OCaml 4.00 introduced the feature of \textit{first-class modules}. A
first-class module is analogous to a language having first-cass
functions: a module in OCaml can be packaged into a value, passed to
functions, and returned from functions. First-class modules provide a
similar capability as functors do, but are much more
light-weight. Refactoring a module into a functor can be a significant
amount of work, but adding a module arguement to a function is a much
simpler change.

%% XXX first class module example
\begin{lstlisting}
module type DATABASE = sig ... end

module Db : DATABASE
\end{lstlisting}

\subsection{The rest of OCaml}

Briefly mention OCaml's object system and class language, and point
the reader to books and web resources.

\section{Test Double patterns in detail}
\label{testdoubles}

\begin{figure}
  \centering
  \includegraphics[scale=1.0]{img/XXX.png}
  \caption[Taxonomy of Test Double patterns]{Taxonomy of Test Double patterns\footnotemark}
  \label{fig:taxonomy}
\end{figure}
\footnotetext{Adapted from Meszaros \cite{meszaros:xunit}}

As figure \ref{fig:taxonomy} shows\dots

XXX disclaimer about the term ``Object'' when referring to test double
patterns, and how we don't necessarily mean object in the OOP sense,
because we can still use these patterns in a functional langauge
without objects.

%% XXX We will want to include diagrams relating all the test double
%% patterns, and describing them individually. The xunit web page has
%% many of these diagrams electronically, so we should use those where
%% we can, with proper attribution.

% http://xunitpatterns.com/Mocks,%20Fakes,%20Stubs%20and%20Dummies.html

\subsection{Dummy Object pattern}
\label{testdoubles:dummy}

(XXX lots of work!) A Dummy Object is used in place of a DOC when the
SUT requires a dependency for it's state, but the SUT doesn't use this
dependency during the test. It may be expensive to create a real
instance of the DOC, so a Dummy Object is created in its place. In
languages like Java, the \code{null} object may be used as a Dummy
Object. (What could we use in OCaml? Perhaps refactor so that the DOC
is an option type? OCaml types are often less complex than Java
objects, so we may have a simple constructor we can use instead of
null. A pattern we often use w.r.t. record types is to construct an
``empty'' value of that type for later extension, so we could reuse
this value for testing purposes. Other than that, perhaps for abstract
types hidden within modules, we may have to provide a ``dummy
instantiator'' in that module (which is also a form of refactoring for
testability).)

%% (XXX) code example!

\subsection{Fake Object pattern}
\label{testdoubles:fake}

(XXX lots of work!) A Fake Object is similar to a Dummy Object. While
a Dummy Object may be as simple as using \code{null} in a language
with nullable types, sometimes a more full implementation is
required. A Fake Object is one which implements the interface of the
DOC, but provides a simpler implementation. For instance, a database
may be replaced by a simple hash table.

%% (XXX) code example!

\subsection{Test Stub pattern}
\label{testdoubles:stub}

(XXX) The Test Stub pattern is the first of the Test Double patterns
which provides functionality specifically for the test case it is used
for. A Test Stub is an implementation of the DOC's interface which can
be programmed to provide set outputs when used by the SUT during the
test. This mechanism is used by the test case to provide indirect
inputs to the SUT via the DOC.

%% (XXX) code example!

\subsection{Test Spy Object pattern}
\label{testdoubles:spy}

(XXX) The Test Spy is used to to replace the DOC in the SUT and record
the interactions that the SUT makes with the DOC. It is analogus to
the the Test Stub, but it works in the opposite way: instead of
providing indirect inputs from the test case to the SUT via the DOC,
it provides indirect outputs from the SUT to the test case via the
DOC.

\subsection{Mock Object pattern}
\label{testdoubles:mocks}

(XXX) Basically combines Test Stub and Test Spy patterns in order to
provide indirect inputs to the SUT, as well as verify the indirect
outputs from the SUT.

\section{Tools for software testing in functional langauges}
\label{testtools}

\begin{enumerate}
\item xUnit implementations: HUnit \cite{www:hunit}, OUnit \cite{www:ounit}

Unit testing frameworks for many languages follow the xUnit pattern of
testing (\cite{junit} \cite{www:nunit} \cite{www:ruby:unit}), and
languages like Haskell and OCaml are no different \cite{www:hunit}
\cite{www:ounit}.

\item Kaputt for OCaml \cite{www:kaputt} (different than OUnit?)

\item Haskell quickcheck \cite{claessen:quickcheck}
\item Haskell benchmarking lib? \cite{www:criterion}
\end{enumerate}
