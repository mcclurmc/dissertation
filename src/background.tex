\chapter{Background}
\label{background}

In this section we will provide a brief tour of the OCaml programming
language, focusing espectially on its powerful module system. The
chapter follows with an in-depth discussion of xUnit's Test Double
patterns and their implementation in OCaml and Java. The chapter
concludes with a summary of the current state-of-the-art testing tools
for functional languages such as OCaml and Haskell.

\section{A brief tour of OCaml and its module system}
\label{ocaml}

% \cite{rwo} \cite{ocaml:spec}

Want to talk quickly about code (functions, types, type inference),
modules, functors, first class modules. Mention that OCaml has a class
language, but that idiomatic OCaml typically uses functions and
modules unless a particular algorithm or solution to a problem
warrants an object-oriented implemtation.

\subsubsection{OCaml basics: values, functions, and types}

OCaml is a strongly, statically typed functional programming
language. It may be interpreted in a read-eval-print loop called the
OCaml ``toplevel,'' or it may be compiled to an OCaml specific
bytecode languate or to a native executable.\cite{ocaml:spec} It has a
very good generational garbage collector tuned for sequential
programs.\cite{ocaml:gc_tutorial}

\subsubsection{The OCaml module system}

\subsubsection{The rest of OCaml}

Briefly mention OCaml's obect system and class language, and point the
reader to books and web resources.

\section{Test Double patterns in detail}
\label{testdoubles}

% \section{The test double pattern, especially mocks}
\subsection{Dummy Object pattern}
\label{testdoubles:dummy}

\subsection{Fake Object pattern}
\label{testdoubles:fake}

\subsection{Test Stub pattern}
\label{testdoubles:stub}

\subsection{Test Spy Object pattern}
\label{testdoubles:spy}

\subsection{Mock Object pattern}
\label{testdoubles:mocks}

\section{Tools for software testing in functional langauges}
\label{testtools}

\begin{enumerate}
\item Haskell quickcheck
\item xUnit implementations: HUnit, OUnit
\item Kaputt for OCaml
\item Haskell benchmarking lib?
\end{enumerate}
