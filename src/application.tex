\chapter{Application}
\label{application}

We will now discuss the application of the Mock Object pattern to the
OCaml programming language. In this chapter we will cover dependency
injection techniques in OCaml, the manual creation of a mock module
and expectations, a domain specific language for specifying
expectations in OCaml, and the automatic generation of mock modules
from a module interface.

\section{Dependency injection methods in OCaml}
\label{application:di}

Dependency injection (DI) techniques in object oriented languages
typically focus on lifting hard-coded dependencies into a class's
constructor so that dependencies can be injected at object creation
time. In OCaml things are not much different, except that we will use
language features such as functors and first-class modules for
injecting module dependencies. Complex functionality typically found
in more mature DI frameworks is not necessary for the rest of this
work, and we will not describe their implementation here.

\section{Manually-written Mock Modules}
\label{application:manual-mock}

Before we can automatically generate a mock module from a given
interface, we must first describe how to create manually-written
(hard-coded) mock modules.

\section{Specifying a Mock's expectations}
\label{application:expectations}

Much of the Mock Pattern's power comes from having a lightweight
expectation specification DSL. In this section we describe the design
and implementation of an expectation DSL in OCaml which is similar to
the JMock expectation DSL described in section
\ref{testdoubles:mocks}.

\section{Automatically generating Mock Modules}
\label{application:generation}

Mocks are only truly useful as a testing tool when they can be
automatically generated from an interface. We will use Camlp4, an
OCaml preprocessing tool, a new OCaml feature called \textit{extension
  points} used for AST annotation, and the OCaml compiler's
\textit{compiler-libs} in order to generate and pretty-print OCaml
code.

\section{Comparison of mocking frameworks}
\label{application:comparison}

Compare MoCaml, manual mocking with Kaputt, and JMock. Demonstrate
pros and cons of each framework. Should this go in the Reflections
chapter? Probably.
