\chapter{Application}
\label{application}

We will now discuss the application of the Mock Object pattern to the
OCaml programming language. In this chapter we will cover some
preliminary techniques, such as dependency injection in OCaml, and the
manual creation of a mock module and expectations. We will move on to
discuss a domain specific language for specifying expectations of a
mock module, and the automatic generation of mock modules from a
module interface.

\section{Preliminary discussion}

\subsection{Dependency injection methods in OCaml}
\label{application:di}

Dependency injection (DI) techniques in object oriented languages
typically focus on lifting hard-coded dependencies into a class's
constructor so that dependencies can be injected at object creation
time. In OCaml things are not much different, except that we will use
language features such as functors and first-class modules for
injecting module dependencies. Complex functionality typically found
in more mature DI frameworks is not necessary for the rest of this
work, and we will not describe their implementation here.

%% \lstinputlisting[
%%     caption={Example of dependency injection techniques in OCaml},
%%     label=code:di,
%%     aboveskip=\baselineskip,
%%   ] {code/application/depinj.ml}

\lstinputlisting[
    caption={Example of basic DI techniques in OCaml},
    label=code:di_basic,
    aboveskip=\baselineskip,
  ] {code/application/depinj_basic_functor.ml}

In listing \ref{code:di_basic} we see a basic DI technique. We
``lift'' the SUT into a functor which receives the DOC. We can use
this method to specialise the SUT with either the ``production'' or
``test'' DOC at compile time.

There are downsides to DI via functorisation, however. In order to
lift the SUT, we have to do significant refactoring -- especially if
the module in question is at the top-level of a compilation unit,
because ml files cannot be functors.

A pattern for lifting a top-level module into a functor is to surround
the whole file with a new module called \code{Make}:
\code{module Make (D : DOC) = struct ... }, and at the end of the file,
after we end the Make module, add \code{include Make(DOC)}, which will
modify this module so that it contains both a functor to create a new
module, and the original contents of the module. If the ml file is
restricted by an mli file, it may be simpler to just move the functor
to a new ml file (say, ``SUT\_func.ml''), and have the original ml
file contain only the line \code{include SUT_func.Make(DOC)}.

\lstinputlisting[
    caption={Example of DI in OCaml using first-class modules},
    label=code:di_fcm,
    aboveskip=\baselineskip,
] {code/application/depinj_fcm.ml}

Alternatively, instead of using functors, we could use first-class
modules to perform dependency injection. We have a couple different
options at our disposal when using first-class modules. Our first
option is to modify each function in the SUT which depends on the DOC
so that it receives the DOC as a first-class module. This is shown in
function \code{g} of listing \ref{code:di_fcm}. We make this an
optional argument so that we don't have to modify the callers.

Modifying the parameter list of each of the SUT's functions may not be
practical. Instead, our second option is to create a first-class
module reference, which points to the current DOC. The caller of the
SUT would set this reference before executing the SUT's
functions. Each function must then be modified to reference the DOC
through the first-class module.

These two methods using first-class modules are best used when
modifying existing code for dependency injection, when the SUT being
modified makes use of the DOC in only a few places. If use of the DOC
in the SUT is pervasive, full module lifting may be a better option.

\lstinputlisting[
    caption={Example of a DI ``Factory'' in OCaml},
    label=code:di_factory,
    aboveskip=\baselineskip,
  ] {code/application/depinj_factory.ml}

Another issue with the above first-class module DI technique is that
it pushes the choice of which DOC to use, either the production module
or the test module, down into the caller of the SUT. We may prefer to
make that decision elsewhere, so that neither the SUT nor the SUT's
callers need to know what DOC is being used.

We can achieve this goal with first-class modules using a ``module
factory,'' as in listing \ref{code:di_factory}. This is analogous to a
``factory'' in the object-oriented sense. Client modules would ask
this factory to create a particular module, and the module factory
would check whether it should return the production dependency or the
test dependency, depending on perhaps a command-line flag or some
configuration file. A major benefit to this method is that it removes
the need to refactor each of the functions that access the first class
module, because we can set this module at module initialisation time.

\lstinputlisting[
    caption={Usage of various DI techniques in OCaml},
    label=code:di_useage,
    aboveskip=\baselineskip,
  ] {code/application/depinj_usage.ml}

%% \section{Design considerations for mock modules}
%% \label{application:manual-mock}

\subsection{An example: the simplest mock module implementation}
\label{application:simple}

Before we can automatically generate a mock module from a given
interface, we must first describe how to create manually-written
(hard-coded) mock modules.

Note that this method does not provide any features we would expect
from a full-featured mocking framework such as JMock. There is no
automatic generation of the mock implementation. There is no
expectation language, or even a means to specify expectations. And
because there is no way to specify expectations, the test case must
set up the mocked functions manually, and verify the behaviour of the
SUT manually.

While this is a very simplistic method for creating a mock module, we
will continually extend this method in the following sections, until
we have automatically generated mocks with an expression
language.

% XXX Do we really need TURTLE? Why don't we just use SIMPLE instead?

Figure \ref{code:mock_real} shows an implementation of a module which
we will mock throughout the next few sections. This module is called
\code{Turtle}, and describes a simple drawing program which is meant
to be reminiscent of the educational programming language
Logo.\footnote{\url{http://en.wikipedia.org/wiki/Logo_(programming_language)}}
We will treat the \code{Turtle} module as our DOC which we wish to
mock.

\lstinputlisting[
caption={An example module that we wish to mock},
label=code:mock_real,
aboveskip=\baselineskip,
] {code/application/simplest_mock_real.ml}

We see from the \code{TURTLE} signature that the \code{Turtle} module
implements a simple interface. There are functions to make a new
turtle, to turn the turtle $90^o$ in either direction, to move the
turtle forward by a given number of units, and to put the pen up or
down. When the pen is in the down position, moving the turtle would
draw to the screen. We have included a simple combinator operator
\code{>>} which allows us to easily chain together turtle operations.

Figure \ref{code:mock_usage} shows a short program which demonstrates
how one might use the \code{Turtle} library.

\lstinputlisting[
caption={Example usage of the \code{Turtle} library},
label=code:mock_usage,
aboveskip=\baselineskip,
] {code/application/simplest_mock_usage.ml}

Suppose now that we wish to test a function or module which depends
upon this \code{Turtle} library. We may wish to test that this system,
say, calls the \code{turn} function a certain number of times. We can
create a manually mocked implementation of the \code{TURTLE} interface
which replaces the original \code{turn} function with one which records
the number of function invocations, the arguments of the function
calls, and a pre-set value for the function to return upon
invocation. Figure \ref{code:mock_partial} shows such an
implementation of a partial, manual mock of the \code{Turtle} library.

%% Nodes from simplest_mock_partial.ml:
%%
%% A partial mock of TURTLE (only mock's turn function). This works by
%% including the to-be-mocked module in the mock itself, and
%% reimplementing the functions we want to mock. This won't work if
%% initialising the to-be-mocked module will have unwanted side
%% effects.

\lstinputlisting[
caption={A manually mocked OCaml module},
label=code:mock_partial,
aboveskip=\baselineskip,
] {code/application/simplest_mock_partial.ml}

We can use this partial mock by installing it using any of the
dependency injection techniques described section
\ref{application:di}, and then write our test using the newly mocked
DOC. Manually mocking a module in this way might be an effective way
to write a test that would otherwise be impossible, but manually
mocking modules in this manner is tedious and cumbersome. We would much
prefer a method which allows us to describe the way that we expect the
depended upon component to behave when it is interacting with the test
case, and then generate a mock module from that set of expectations.

%\section{Specifying a mock's expectations}
\section{Implementation of the mocking framework}
\label{application:implementation}

%% Synopsis

Much of the Mock Pattern's power comes from having a lightweight
expectation specification language. This language allows users to
describe their expectations of how the mock module will behave in a
natural way. Expectations are then compiled into the underlying data
type representation. The mock module, while being exercised by the
system under test, records its interactions and also plays back the
indirect outputs recorded in the expectations. Finally, the
expectations are verified by the test framework by checking that the
recorded interactions with the mocked module match what was specified
by the user in the expectations.

In this section we will cover the implementation of a mocking
framework. We will discuss the syntax and implementation of the
expectation language, the verification of a mock's expectations, and
the automatic generation of a mock module through compile-time
meta-programming.

\subsection{Syntax of the expectation language}
\label{application:syntax}

The expectation language is made up of a small number of operations
which describe how we expect a mock to behave during a test. These
functions each return a expectation, which can be strung together into
a list of all expectations for that mock.

%% TODO make sure that we're describing the correct syntax (poly vars, etc.)
\begin{figure}
  \begin{grammar}
    <int> :: An integer

    <value> :: An OCaml value

    <function> :: An OCaml function

    <mocked\_function> ::= A function exposed by the mocked module

    <tag> ::= A polymorphic variant that names a mocked function

    <expect\_op> ::=
    `one\_of' <mocked\_function>
    \alt `never' <mocked\_function>
    \alt `times' <int> <mocked\_function>
    \alt `before' <mocked\_function> <mocked\_function>
    \alt <mocked\_function> `>>' <mocked\_function>
    \alt `will' <mocked\_function> (<tag> <function>)
    \alt  <mocked\_function> `*>' (<tag> <function>)
    \alt `returns' <mocked\_function> (<tag> <value>)
    \alt <mocked\_function> `*->' (<tag> <value>)

    <expectations> ::= `[' <expect\_stmt> ( `;' <expect\_stmt> )* `]'
  \end{grammar}

  \caption{Grammar of our expectation language}
  \label{code:syntax}
\end{figure}

In figure \ref{code:syntax} we describe the grammar of our expectation
language. We have a number of expectation operations (\code{one_of},
\code{will}, etc.) which allow a test writer to describe a single
expectation. These expectation operations are grouped together in a
list of expectations, which will be compiled in a later step (see
section \ref{application:mock_impl}). The expectation operations are
explained as follows:

\begin{itemize}
\item \code{one_of f}: At least one invocation of \code{f} ins
  expected.
\item \code{never f}: \code{f} should never be invoked.
\item \code{times n f}: \code{f} should be invoked exactly $n$
  times.
\item \code{before f g}: \code{f} should be invoked at least once
  before \code{g}.
\item \code{f >> g}: Shorthand for \code{before}
\item \code{will f f'}: The mock function \code{f} will behave as
  instead as function \code{f'}.
\item \code{f *> f'}: Shorthand for \code{will}
\item \code{returns f r}: the mock function \code{f} will return the
  value \code{g}, regardless of its inputs during test execution.
\item \code{f *-> r}: Shorthand for \code{returns}
\end{itemize}

%% TODO: use file expect_example.ml, but restrict with firstline/lastline
\begin{figure}
\lstinputlisting[
    caption={Example of expectations list},
    label=code:expect_example,
    aboveskip=\baselineskip,
    numbers=left,
  ] {code/application/expect_example_expectation_list.ml}
\end{figure}

Figure \ref{code:expect_example} shows how one might write an
expectation list. The \code{let e} on line 19 in this listing defines
a list of expectations which are generated by the function calls in
the list. For example, \code{one_of add} will generate the expectation
\code{One_of `add}. The expectation \code{add >> incr} (read as ``add
then incr'') states that we expect first function \code{add} and then
function \code{incr} to be executed, in that order, during the test
case.  \code{incr *> `incr (fun x -> x+2)} (read as ``incr will behave
as ...'')  states that the \code{incr} function will behave as the
function \code{fun x -> x+2} during the test case.
\code{add *-> (`add 42)} (read as ``add returns 42'') states that the
\code{add} function will always return $42$ during test execution,
regardless of input.

The reader will note that some of these operations are a bit
redundant. For instance, in order to specify that the \code{incr}
function should always return the value 42, we must write
\code{incr *-> `incr 42}. The name \code{incr} is repeated twice, the
second time as a polymorphic variant. The reason for this redundancy
is that we must be able to write a \code{returns} function that takes
a function name, and also a variably-typed argument for the return
value (and similar for the \code{will} operation). If we wish to write
both, say, \code{f *-> 42} and \code{g *-> "hello world"}, we would
essentially need to make \code{*->} dependently typed, where the type
of the second argument to \code{*->} depends on the value of the first
\cite[Chapter 2]{pierce:atapl}.

There are a few possibilities for resolving this redundancy. We may
attempt to use OCaml's generalised algebraic data types (GADT) to
provide a ``light-weight dependent type'' to represent the action
operations, but formulating this type of GADT is notoriously
tricky. Instead of pushing the boundaries of the type system to
provide a nice, generic \code{returns} function, we could instead
generate a different \code{returns} and \code{will} functions for each
mocked function (so we would have \code{f_returns}, \code{f_will},
\code{g_returns}, \code{g_will}, etc. for all mocked functions
\code{f}, \code{g}, ...). This solution is quite inelegant, as it
requires generating quite a few redundant-sounding function names. A
third possible solution is to retreat from the confines on the OCaml
type system, and to further use OCaml's pre-processing capabilities to
simply rewrite the natural \code{f *> (fun ...)} to the more
cumbersome \code{f *> (`f (fun ...))}.\footnote{There exists a fourth
  solution which would allow test writers to write more elegant
  expectations, at the expense of distasteful coding style on the part
  of the library writer. OCaml provides the \code{Obj.magic}
  function, which has the type \code{Obj.magic : a -> b}. (It is akin
  to Haskell's similarly distasteful \code{unsafeCoerce}.) While
  this function certainly has its uses, it is generally recommended to
  avoid its use at all costs.}

%% XXX TODO This subsection on syntax needs work. We need to make sure
%% that the grammar shows where we have to insert poly variants, and then
%% explain why those are necessary. Move-up the listing on the generated
%% types, and explain what they do. Discuss possible options for using
%% GADTs or other interesting types to represent actions so that we have
%% a cleaner syntax, and also discuss the possibility of doing AST
%% rewriting on the expectation definition, generating \code{<fun>_will}
%% functions specific for each mocked function, or using extension points
%% for the action definitions, in order to clean up the syntax. Something
%% like:

%% \begin{lstlisting}[aboveskip=\baselineskip,
%%   caption=Extension points for cleaner expecation syntax]
%%   let expectations = [
%%     [% fun_1 returns 42 %];
%%     [% fun_2 will (fun x -> ...) %];
%%     [% one_of fun_3 %]
%%   ]
%% \end{lstlisting}

%% XXX This point about indirect inputs is good.
%% \textit{Some considerations when implementing
%%   expectations. Expectations are used for two purposes: 1) to allow
%%   the mock implementation to provide the appropriate indirect inputs to
%%   the SUT (through the expectation's actions), and 2) to provide
%%   indirect outputs to the test case from the SUT, in the form of
%%   invocation counts, as well as values passed to the mocked
%%   functions.}

\subsection{Implementation of the expectation language}
\label{application:mock_impl}

%% XXX needs more exposition

The implementation of the mock expectation language follows these
steps:

\begin{enumerate}
\item Generate a mock from a specified module
\item Describe the expectations
\item Compile the expectations
\item Execute the test in the context of the compiled expectations
\item Verify the expectations
\end{enumerate}

In the previous section we covered the second step, describing the
grammar of the expectation language. The expectation language is meant
for ease of use by the test writer, as well as for ease of reading,
for the benefit of any future programmers attempting to understand the
test case. This list of expectations as written by the test writer is
not helpful to us when verifying that the mock's expected behaviour was
exhibited by the test case. We must now compile the given expectations
into a format that can be used by the mock framework to verify the
behaviour of the mock.

Expectations can be grouped into three categories: invocation
counting, ordering, and actions. Invocation counting expectations set
bounds on the number of times a particular function may be called
during a test. Ordering expectations enforce a partial order on
function invocations. Action expectations state that a particular
function will behave in a certain way upon invocation, either by
returning a particular value, or by invoking a completely different,
test-writer specified, function. The format to which we compile our
expectations must support the verification of these types of
behaviours.

% TODO explain these two listings

\begin{lstlisting}[aboveskip=\baselineskip,
    caption=Example of types needed for a generated mock module,
    label=code:simple_mock_types]
(* Mocked function names *)
type fname = [ `add | `incr ]
(* Function types *)
type ftype =
  [ `add of (int -> int -> int)
  | `incr of (int -> int) ]
(* Function arg types *)
type farg =
  [ `add of (int * int)
  | `incr of int ]
(* Function return types *)
type fret =
  [ `add of int
  | `incr of int ]
(* "Action" types (these are *not* generated!) *)
type fact =
  [ `f_do of ftype
  | `f_ret of fret ]
\end{lstlisting}

\begin{lstlisting}[aboveskip=\baselineskip,
    caption=The expectation type,
    label=code:expect_type]
type exp =

  (* Counting *)
  | One_of of fname
  | Never of fname
  | Times of (int * fname)

  (* Order *)
  | Before of (fname * fname)

  (* Action *)
  | Will of (fname * ftype)
  | Returns of (fname * fret)
\end{lstlisting}

\subsubsection{Invocation counting expectations}

In order to verify invocation counting expectations, it suffices to
provide a means for the verification system to query both the actual,
and expected, invocation counts for a particular function. We can
provide this functionality by encapsulating (function name, invocation
count) tuples as an OCaml \code{Set}. As part of the mock generation
process, we generate a set of polymorphic variants of the same name as
the functions of the mocked module. During the test execution, we keep
track of the number of times a function is executed by wrapping it in
a ``proxy'' function, which updates the execution count of that
function each time it is called.

\subsubsection{Order expectations}

Order expectations specify a partial order on the execution of
particular functions. We do wish to burden the test writer by forcing
him or her to specify a total order on all of the function invocations
in the test, so we have defined a simple \code{before f g} expectation
which enforces that \code{f} must be invoked at least once before
\code{g}.

For instance, say we have a mock with three functions, \code{f},
\code{g}, and \code{h}. We may expect that \code{f} occurs before
\code{g}; in this case we do not care where in the execution order
function \code{h} appears, nor do we care when the execution order is
\code{f;g;f}, because \code{f} has already occurred before \code{g}.

In order to verify an order expectation, we need a way to determine
whether a function's expected predecessors have been executed prior to
the given function. To do this, we compile the set of a function's
``dominators'', or those functions which the test writer has stated
that we should expect to be executed before the given function. During
execution of the test, we then keep track of execution order of the
mocked functions. At each function invocation, we verify that the
given function's dominators have each appeared in the execution order
prior to this function's invocation.

\subsubsection{Action expectations}

Action expectations are particularly interesting because they serve
two purposes: 1) to allow the test writer to verify that a function is
invoked with particular arguments, and 2) to make it possible to test
hard-to-test code by allowing the test writer to ``mock-out'' the
behaviour of the original function, which may not be possible to
perform during a test case. So action expectations actually perform
double duty as expectation verifiers and as test-case facilitators.

%% (XXX how do we verify this? A set of function name and function
%% actions. The generated proxy for this function will test whether there
%% exists an action defined for it, and will perform this action in place
%% of the original function, if the action expectation exists. An action
%% expectation may choose to throw an \code{Expect_Violation} exception
%% if it was invoked with incorrect arguments.)

In order to verify an action expectation, we must allow the test
writer to specify an action that will occur in place of the original
function during the execution of the test. We allow the test writer to
specify either a function to replace the mocked function, or a return
value that will be returned by the mocked function, regardless of the
arguments with which the mocked function is invoked in the test
case. We encode action expectations as a variant type of either
\code{`f_do} (for function actions) or \code{`f_ret}, for return
actions. (See listing \ref{code:simple_mock_types} for the definition
of these variants.)


%% \subsection{Implementation of mock module implementing the expression langauge}
%% \subsubsection{Module organisation}

%% We will be generating a lot of code when we generate mock
%% modules. We'll have to generate an implementation of the to-be-mocked
%% interface. We'll have to generate expectation language code which uses
%% the same types as defined in the to-be-mocked interface, such that
%% the types in the expectation langauge will unify with those in the
%% mocked module. We'll have to generate code to verify the expectations
%% that have been defined in the test case.

%% The organisation of generated modules must be chosen carfully,
%% especially to satisfy the requirement that types between the
%% expectation langauge and the mocked module must unify.

%% One solution to this is to put logically separate code into separate
%% modules, and strategically include modules in ``derivative'' modules
%% where necessary. (This is what we have currently done.) Doing this
%% allows us to properly namespace the identifiers that we have
%% generated, so that we don't have to mangle generated names and then
%% redefine them later.

%% Another solution is to place all generated code into a single module
%% that is not exported, but to then create separate modules (say
%% \code{E} for the expecation language and \code{M} for the mocked
%% module), where each of these modules is restriced by an appropriate
%% signature. This should give us the most flexibility: no cumbersome
%% module inclusions, while hopefully retaining types that are
%% unifiable. Unfortunately, with this solution we will have to mangle
%% our generated names and then redefine them later, because we may clash
%% with the names in our to-be-mocked interface. We can minimise the
%% names we have to mangle by keeping track of the identifiers defined in
%% the to-be-mocked module, and only mangling clashing names. On second
%% thought, this sounds far too complex; more complex than managing
%% multiple modules.

%% \subsubsection{Generalised Algebraic Data Types}
%% \label{application:gadt}

%% We make use of Generalised Algebraic Data Types (GADTs) to help us
%% encode the arguements for and return values of our expectations. GADTs
%% are a relatively new feature in OCaml. They allow \dots

% (Also cite Ralf's Phantom Types paper! :) )

\subsection{Verifying a mock's expectations}
\label{application:mock_verification}

Now that we have described the compiled form of the expectations, we
may now describe how the mock framework verifies that the given
expectations have been satisfied during the execution of the test.

Verification of expectations proceeds in two phases. The first phase
occurs during the actual execution of the test case. Most expectations
can ``short-circuit,'' and raise an \code{Expect_Violation} exception
during the test case execution. For instance, if we expect that
function \code{f} will never be invoked, and at some point in the test
case that function is invoked, we can immediately raise an error.

On the other hand, some expectations must wait until the end of the
test case to be verified. For example, consider the \code{never f}
expectation earlier. What if function \code{f} is never invoked? We
need to tell the test framework that the test is over in order to make
the ultimate decision about whether all the expectations have been
met. Consider also an expectation of \code{one_of g}, where we expect
function \code{g} to be invoked at least once during testing. If the
test case never invokes \code{g}, then the mock framework won't be
able to verify whether this expectation has been met until it knows
that the test case has finished. In order to tell the mock framework
that the test is over, the test writer must execute the mock's
\code{verify} function, which re-verifies all of the compiled
expectations, and raises an \code{Expect_Violation} exception if any
have not been satisfied. This is the second phase of expectation
verification.

For the mock framework to be able to verify expectations at all, it
must know when and how each of the mocked functions are invoked. To
that end, we wrap each of the mocked functions in a ``proxy''
function, which records information about the invocation, performs any
possible first-phase verifications, and then performs the appropriate
action: either execute the original function, return the expected
value, or execute the expected function.

\begin{lstlisting}[aboveskip=\baselineskip,
    caption=A proxy for a mocked function,
    numbers=left,
    label=code:simple_mock_add_proxy]
let add_proxy m a b =
  (* Record this function invocation *)
  m.fcalls <- m.fcalls @ [`add, `add (a, b)];
  if ExpectMap.mem `add m.expects
  then
    let ncalls = List.(filter
                         (fun (f,_) -> f = `add)
                         m.fcalls
                       |> length) in
    let e = ExpectMap.find `add m.expects in

    (* Test for a violated exact count expectation *)
    (match e.count with
    | Some (Exactly count) ->
       if ncalls > count
       then raise (Expect_violation
                     (`add (a, b),
                      Count_violation ((Exactly count), ncalls)));
    | _ -> ());

    (* Test for a violated order expectation *)
    FSet.iter
      (fun d ->
       if not (List.exists (fun (f,_) -> f = d) m.fcalls)
       then raise (Expect_violation (`add (a, b), Order_violation d)))
      e.dominators;

    (* Perform any actions *)
    match e.action with
    | Some (`f_do f)  -> (match f with
                          | `add f' -> f' a b
                          | _ -> failwith "Impossible")
    | Some (`f_ret x) -> (match x with
                          | `add x' -> x'
                          | _ -> failwith "Impossible")
    | None -> Simple.add a b
  else Simple.add a b
\end{lstlisting}

We will now discuss the process of verifying each type of expectation.

\subsubsection{Verifying invocation count expectations}

Verification of an expectations invocation count is trivial. We simply
need to ensure that any exact count expecations are not exceeded at
the point of the mock function invocation. For any ``at most'' or ``at
least'' count expectations, we must wait until the second phase before
we can verify. In this phase, we simply iterate over all of the
invocation count expectations, and ensure that the actual invocation
count satisfies these expectations.

\subsubsection{Verifying order expectations}

Order expectations are likewise straight forward. The best time at
which to verify order violations is in the first phase, during the
initial test execution. Listing \ref{code:simple_mock_add_proxy},
starting on line 21, demonstrates how this is achieved. We retrieve
the set of dominators of the most recently invoked function, and we
test that each of these dominators appears in the list of previous
function invocations. We could still perform this verification in the
second phase of verification, if we desire. In that case, we would
instead iterate over the prefixes of the function invocation list, and
verify the dominators of each of the last function to be invoked in
the current prefix list.

\subsubsection{Verifying action expectations}

Action expectations are similarly verified in the first phase. Action
expectations which only specify a return value are not necessary to be
verified, because their sole purpose is to provide indirect inputs to
the test case. Action expectations which provide an alternative
function definition for the mocked function are in fact
self-verifying. The test writer may choose to raise an
\code{Expect_violation} exception upon receiving invalid input. This
is equivalent to the JMock library's ability to set expectations on
function arguments, but it instead takes advantage of OCaml's anonymous
functions and pattern matching. Listing \ref{code:verify_action}
demonstrates this capability with an expectation which raises an error
when \code{incr} is called with values $2$ or $3$.

\begin{lstlisting}[aboveskip=\baselineskip,
    caption=An action expectation example,
    label=code:verify_action]
let expectations = [
  will `incr (function
              | 2
              | 3 -> raise (Expect_violation
                            (Violation_msg
                             "2 and 3 are terrible numbers...")))
  ]
\end{lstlisting}

%% To verify expectations, we need to compare them to the invocation
%% record, and ensure that the invocations satisfy the expecatations. We
%% need to ensure:

%% \begin{enumerate}
%% \item All invocation counts are satisfied. This includes insuring that
%%   \textit{never} functions are not called, and that number-checked
%%   functions are called the correct number of times. We can skip
%%   testing for \textit{allowing} functions, since we don't care whether
%%   they are called.
%% \item Sequenced expectations are invoked in the order
%%   specified. (We'll do this later, not as a first feature.)
%% \end{enumerate}

%% To verify expectations, we need to iterate over each invocation, and
%% imperatively update the list of expectations accordingly. We search
%% for the current invocation in the list of expectations, and increment
%% the invocation count. After we've interated through the invocations,
%% we have an updated list of expectations that we can then verify
%% (short-cut: we update the list of expectations at test-time, instead
%% of recording invocations). Now we proceed through the expectations and
%% verify that the invocation count satisfies the expectation. Later,
%% when we implement sequences, we can also verify that the order of
%% invocations is correct.\footnote{Aha! We can simply attach an
%%   invocation sequence number to each expectation as we invock a
%%   matching function. This will allow us to verify that the
%%   expectations in each sequence have a monotonically increasing
%%   invocation sequence number.}

%% We should be sure to proceed through the entire list of expectations,
%% and not necessarily short-circuit at the first matching
%% expectation. If that first matching expectation has already been
%% satisfied, then we will need to proceed to the next potentially
%% matching expectation. This means that we will need to perform
%% verification at each invocation step, so that we can easily skip
%% already-satisifed expectations.

\subsection{Automatic generation of a mock module}
\label{application:generation}

Mocks are only truly useful as a testing tool when they can be
automatically generated. We will use a new feature of the OCaml 4.02
compiler called ``extension points.'' \cite{ocaml:spec}. Extension
points are a replacement for OCaml's previous pre-processing tool
called Camlp4. Extension points provide a means for manipulating the
abstract syntax tree (AST) of a program before compilation is
finished. They are similar to Lisp's macro system, except that they
are much more low-level, in that there is no anti-quotation system, and
programmers must directly manipulate the compiler's view of the AST
during the compilation of the program. In addition to providing a
means to manipulate the AST of a program directly, extension points
also provide a way to annotate source code. Annotations essentially
add non-semantic nodes to the AST, which the compiler will ignore. A
extension, however, can use this extra AST node however it wishes.

We will make use of these extension points when generating mock
modules.

%% \begin{lstlisting}[aboveskip=\baselineskip,
%%     caption=Example of using the \code{mock} annotation,
%%     label=code:mock_annotation]
%%   (* Some module that we wish to mock *)
%%   module DOC = struct ... end

%%   (* We must extract the signature of the module *)
%%   module type DOC_S = module type of DOC

%%   (* Use an annotation to tell the MoCaml extension to
%%      generate a mock for the DOC_S interface. *)
%%   module DOC_mock = [% mock DOC_S %]
%% \end{lstlisting}

%% \textit{XXX show a dumped parse tree of an interface want to mock?}

\begin{lstlisting}[aboveskip=\baselineskip,
    caption={An annotated module, from which a mock module will be generated},
    label=code:simple_annotated]
module type SIMPLE_S =
  sig
    val add  : int -> int -> int
    val incr : int -> int
  end [@mock]
\end{lstlisting}

Listing \ref{code:simple_annotated} shows a module which we have
worked with previously, which has been given the \code{[@mock]}
annotation. Our mock-generating extension uses this annotation to
decide for which module signatures to generate mock module
implementations.

A limitation of using extension points is that we only have the
current abstract syntax tree to work with. Because all extensions are
run before type checking occurs, we do not have the benefit of the
OCaml type checker when generating our mock modules. This means that
while we would like to be able to annotate a module implementation, or
to recover the signature of an existing module and generate a mock
from that, we do not have enough type information at this stage to
generate a well typed mock module.\footnote{As we noted previously,
  the function \code{Obj.magic} would allow us to ``fake'' the
  appropriate types to mock a module implementation, without relying
  on a module type signature. This would allow us to generate a mock
  module containing functions with the same arity as the original
  implementation, but which would have fully polymorphic type
  signatures. This type of mock implementation would be dangerous, as
  it would rely on the test writer to provide correctly-typed values
  to the action expecations. We would prefer to restrict the mock
  framework to generating mocks for fully specified module signatures,
  rather than take the risk of lying to the type system.}

Listing \ref{code:simple_annotated_ast} shows the AST fragment
produced by the module type signature and annotation from listing
\ref{code:simple_annotated}. Without careful study of the OCaml
compiler internals this AST fragment is quite meaningless, but it
serves to show that from the above module type and annotation we can
glean the type of each value declared in the original signature, which
is enough to generate a mock implementation of this signature.

\begin{figure}
\begin{lstlisting}[aboveskip=\baselineskip,
    caption={The AST node produced by \ref{code:simple_annotated}},
    %% numbers=left,
    label=code:simple_annotated_ast]
[{pstr_desc =
   Pstr_modtype
    {pmtd_name = {txt = "SIMPLE_S"};
     pmtd_type =
      Some
       {pmty_desc =
         Pmty_signature
          [{psig_desc =
             Psig_value
              {pval_name = {txt = "add"};
               pval_type =
                {ptyp_desc =
                  Ptyp_arrow ("",
                   {ptyp_desc = Ptyp_constr ({txt = Lident "int"}, [])},
                   {ptyp_desc =
                     Ptyp_arrow ("",
                      {ptyp_desc = Ptyp_constr ({txt = Lident "int"}, [])},
                      {ptyp_desc = Ptyp_constr ({txt = Lident "int"}, [])})})};
               pval_prim = []}};
           {psig_desc =
             Psig_value
              {pval_name = {txt = "incr"};
               pval_type =
                {ptyp_desc =
                  Ptyp_arrow ("",
                   {ptyp_desc = Ptyp_constr ({txt = Lident "int"}, [])},
                   {ptyp_desc = Ptyp_constr ({txt = Lident "int"}, [])})};
               pval_prim = []}}];
        pmty_attributes = [({txt = "mock"}, PStr [])]}}};
\end{lstlisting}
\end{figure}

\begin{figure}
\begin{lstlisting}[aboveskip=\baselineskip,
    caption={Elegant, but invalid, methods of specifying mock module},
    label=code:simple_annotated_bad]
(* 'Simple' module does not have any type information, so
   there is nothing to work with when generating a mock. *)
module Simple = struct
  let add a b = a + b ;;
  let incr a = a + 1;;
end [@mock]

(* We still have no type information, so this extension is
   not useful to us either. *)
module SimpleMock_struct = [%mock Simple]

(* Similarly, a recovered signature is represented in the AST
   as just an identifier that points to the original module
   -- there is no type information available to us. *)
module type SIMPLE = module type of Simple
module SimpleMock_recovered_sig = [%mock SIMPLE]
\end{lstlisting}
\end{figure}

By writing a pre-processor extension which extracts the type
signatures of each of the values of a given module, we can generate
all of the machinery necessary to implement a mock module. This
includes a new module; with nested \code{Mock} and \code{Expect} inner
modules; the type and value definitions required to represent the
values declared in the original interface; the function proxies which
record function invocations and perform verifications; and the
expectation and mock data types, which represent the defined and
compiled expectations. The generated mock module also includes
non-generated functions and types, such as the \code{compile} and
\code{verify} functions, which compile and verify expecations.
