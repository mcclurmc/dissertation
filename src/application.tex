\chapter{Application}
\label{application}

We will now discuss the application of the Mock Object pattern to the
OCaml programming language. In this chapter we will cover dependency
injection techniques in OCaml, the manual creation of a mock module
and expectations, a domain specific language for specifying
expectations in OCaml, and the automatic generation of mock modules
from a module interface.

\section{Dependency injection methods in OCaml}
\label{application:di}

Dependency injection (DI) techniques in object oriented languages
typically focus on lifting hard-coded dependencies into a class's
constructor so that dependencies can be injected at object creation
time. In OCaml things are not much different, except that we will use
language features such as functors and first-class modules for
injecting module dependencies. Complex functionality typically found
in more mature DI frameworks is not necessary for the rest of this
work, and we will not describe their implementation here.

%% \lstinputlisting[
%%     caption={Example of dependency injection techniques in OCaml},
%%     label=code:di,
%%     aboveskip=\baselineskip,
%%   ] {code/application/depinj.ml}

\lstinputlisting[
    caption={Example of basic DI techniques in OCaml},
    label=code:di_basic,
    aboveskip=\baselineskip,
  ] {code/application/depinj_basic_functor.ml}

In listing \ref{code:di_basic} we see a basic DI technique. We
``lift'' the SUT into a functor which receives the DOC. We can use
this method to specialize the SUT with either the ``production'' or
``test'' DOC at compile time.

There are downsides to DI via functorisation, however. In order to
lift the SUT, we have to do significant refactoring -- especially if
the module in question is at the top-level of a compilation unit,
because ml files cannot be functors.

A pattern for lifting a top-level module into a functor is to surround
the whole file with a new module called \code{Make}: \code{module Make
  (D : DOC) = struct ...}, and at the end of the file, after we end
the Make module, add \code{include Make(DOC)}, which will modify this
module so that it contains both a functor to create a new module, and
the original contents of the module. If the ml file is restricted by
an mli file, it may be simpler to just move the functor to a new ml
file (say, ``SUT\_func.ml''), and have the original ml file contain only
the line \code{include SUT_func.Make(DOC)}.

\lstinputlisting[
    caption={Example of DI in OCaml using first-class modules},
    label=code:di_fcm,
    aboveskip=\baselineskip,
] {code/application/depinj_fcm.ml}

Alternatively, instead of using functors, we could use first-class
modules to perform dependency injection. We have a couple different
options at our disposal when using first-class modules. Our first
option is to modify each function in the SUT which depends on the DOC
so that it receives the DOC as a first-class module. This is shown in
function \code{g} of listing \ref{code:di_fcm}. We make this an
optional arguement so that we don't have to modify the callers.

Modifying the parameter list of each of the SUT's functions may not be
practical. Instead, our second option is to create a first-class
module reference, which points to the current DOC. The caller of the
SUT would set this reference before executing the SUT's
functions. Each function must then be modified to reference the DOC
through the first-class module.

These two methods using first-class modules are best used when
modifying existing code for dependency injection, when the SUT being
modified makes use of the DOC in only a few places. If use of the DOC
in the SUT is pervasive, full module lifting may be a better option.

\lstinputlisting[
    caption={Example of a DI ``Factory'' in OCaml},
    label=code:di_factory,
    aboveskip=\baselineskip,
  ] {code/application/depinj_factory.ml}

Another issue with the above first-class module DI technique is that
it pushes the choice of which DOC to use, either the production module
or the test module, down into the caller of the SUT. We may prefer to
make that decision elsewhere, so that neither the SUT nor the SUT's
callers need to know what DOC is being used.

We can acheive this goal with first-class modules using a ``module
factory,'' as in listing \ref{code:di_factory}. This is analogous to a
``factory'' in the object-oriented sense. Client modules would ask
this factory to create a particular module, and the module factory
would check whether it should return the production dependency or the
test dependency, depending on perhaps a command-line flag or some
configuration file. A major benefit to this method is that it removes
the need to refactor each of the functions that access the first class
module, because we can set this module at module initialisation time.

\lstinputlisting[
    caption={Usage of various DI techniques in OCaml},
    label=code:di_useage,
    aboveskip=\baselineskip,
  ] {code/application/depinj_usage.ml}

\section{Design considerations for mock modules}
\label{application:manual-mock}

Before we can automatically generate a mock module from a given
interface, we must first describe how to create manually-written
(hard-coded) mock modules.

\section{An example: the simplest mock module implementation}
\label{application:simple}

Note that this method does not provide any features we would expect
from a full-featured mocking framework such as JMock. There is no
automatic generation of the mock implementation. There is no
expectation language, or even a means to specify expectations. And
because there is no way to specify expectations, the test case must
set up the mocked functions manually, and verify the behaviour of the
SUT manually.

While this is a very simplistic method for creating a mock module, we
will continually extend this method in the following sections, until
we have automatically generated mocks with an expression
language.

Figure \ref{code:mock_real} shows an implementation of a module which
we will mock throughout the next few sections. This module is called
\code{Turtle}, and describes a simple drawing program which is meant
to be reminiscent of the educational programming language
Logo.\footnote{\url{http://en.wikipedia.org/wiki/Logo_(programming_language)}}
We will treat the \code{Turtle} module as our DOC which we wish to
mock.

\lstinputlisting[
caption={An example module that we wish to mock},
label=code:mock_real,
aboveskip=\baselineskip,
] {code/application/simplest_mock_real.ml}

We see from the \code{TURTLE} signature that the \code{Turtle} module
implements a simple interface. There are functions to make a new
turtle, to turn the turtle $90^o$ in either direction, to move the
turtle forward by a given number of units, and to put the pen up or
down. When the pen is in the down position, moving the turtle would
draw to the screen. We have included a simpile combinator operator
\code{>>} which allows us to easily chain together turtle operations.

Figure \ref{code:mock_usage} shows a short program which demonstrates
how one might use the \code{Turtle} library.

\lstinputlisting[
caption={Example usage of the \code{Turtle} library},
label=code:mock_usage,
aboveskip=\baselineskip,
] {code/application/simplest_mock_usage.ml}

Suppose now that we wish to test a function or module which depends
upon this \code{Turtle} library. We may wish to test that this system,
say, calls the \code{turn} function a certain number of times. We can
create a manually mocked implementation of the \code{TURTLE} interface
which replaces the origina \code{turn} function with one which records
the number of function invocations, the arguments of the function
calls, and a pre-set value for the function to return upon
invocation. Figure \ref{code:mock_partial} shows such an
implementation of a partial, manual mock of the \code{Turtle} library.

%% Nodes from simplest_mock_partial.ml:
%%
%% A partial mock of TURTLE (only mock's turn function). This works by
%% including the to-be-mocked module in the mock itself, and
%% reimplementing the functions we want to mock. This won't work if
%% initialising the to-be-mocked module will have unwanted side
%% effects.

\lstinputlisting[
caption={A manually mocked OCaml module},
label=code:mock_partial,
aboveskip=\baselineskip,
] {code/application/simplest_mock_partial.ml}

We can use this partial mock by installing it using any of the
dependency injection techniques described section
\ref{application:di}, and then write our test using the newly mocked
DOC. Manually mocking a module in this way might be an effective way
to write a test that would otherwise be impossible, but manually
mocking modules in this manner is tedius and cumbersome. We would much
prefer a method which allows us to describe the way that we expect the
depended upon component to behave when it is interacting with the test
case, and then generate a mock module from that set of expectations.

\section{Specifying a mock's expectations}
\label{application:expectations}

\textit{Much of the Mock Pattern's power comes from having a
  lightweight expectation specification DSL. In this section we
  describe the design and implementation of an expectation DSL in
  OCaml which is similar to the JMock expectation DSL described in
  section \ref{testdoubles:mocks}.}

\textit{Some considerations when implementing
  expectations. Expectations are used for two purposes: 1) to allow
  the mock implmentation to provide the appropriate indirect inputs to
  the SUT (through the expectation's actions), and 2) to provide
  indirect outputs to the test case from the SUT, in the form of
  invocation counts, as well as values passed to the mocked
  functions.}

\textit{For expectations, we need to keep track of the expected
  invocation count, and the expected outputs of the function, whether
  they are values or expections. When (and if) we implement individual
  sequences, we need to keep track of which expectations are in which
  sequence.}

\textit{While running the test case, the mock implementation needs to
  keep track of the invocations of each function, including the order
  in which they were called, as well as with what arguments they were
  called.}

\textit{Note: a feature that we'd like to have is expectation pattern
  matching. We'd like to be able to set an expectation that a function
  will be called with some pattern-restricted set of arguments, and
  then either verify that the the function was invoked with arguments
  that match the pattern, or simply inspect the arguements passed to
  the function in the test itself. We might want to implement this
  feature by allowing the user to pass arbitrary funtions which match
  the type of the mocked function to the mock as the mock function
  itself, instead of generating an entire function using just the
  invocation arguments and expected return value.}

\subsection{Module organisation}

We will be generating a lot of code when we generate mock
modules. We'll have to generate an implementation of the to-be-mocked
interface. We'll have to generate expectation language code which uses
the same types as defined in the to-be-mocked interface, such that
the types in the expectation langauge will unify with those in the
mocked module. We'll have to generate code to verify the expectations
that have been defined in the test case.

The organisation of generated modules must be chosen carfully,
especially to satisfy the requirement that types between the
expectation langauge and the mocked module must unify.

One solution to this is to put logically separate code into separate
modules, and strategically include modules in ``derivative'' modules
where necessary. (This is what we have currently done.) Doing this
allows us to properly namespace the identifiers that we have
generated, so that we don't have to mangle generated names and then
redefine them later.

Another solution is to place all generated code into a single module
that is not exported, but to then create separate modules (say
\code{E} for the expecation language and \code{M} for the mocked
module), where each of these modules is restriced by an appropriate
signature. This should give us the most flexibility: no cumbersome
module inclusions, while hopefully retaining types that are
unifiable. Unfortunately, with this solution we will have to mangle
our generated names and then redefine them later, because we may clash
with the names in our to-be-mocked interface. We can minimise the
names we have to mangle by keeping track of the identifiers defined in
the to-be-mocked module, and only mangling clashing names. On second
thought, this sounds far too complex; more complex than managing
multiple modules.

\subsection{Generalised Algebraic Data Types}
\label{application:gadt}

We make use of Generalised Algebraic Data Types (GADTs) to help us
encode the arguements for and return values of our expectations. GADTs
are a relatively new feature in OCaml. They allow \dots

% (Also cite Ralf's Phantom Types paper! :) )

\subsection{Verifying expectations}

To verify expectations, we need to compare them to the invocation
record, and ensure that the invocations satisfy the expecatations. We
need to ensure:

\begin{enumerate}
\item All invocation counts are satisfied. This includes insuring that
  \textit{never} functions are not called, and that number-checked
  functions are called the correct number of times. We can skip
  testing for \textit{allowing} functions, since we don't care whether
  they are called.
\item Sequenced expectations are invoked in the order
  specified. (We'll do this later, not as a first feature.)
\end{enumerate}

To verify expectations, we need to iterate over each invocation, and
imperatively update the list of expectations accordingly. We search
for the current invocation in the list of expectations, and increment
the invocation count. After we've interated through the invocations,
we have an updated list of expectations that we can then verify
(short-cut: we update the list of expectations at test-time, instead
of recording invocations). Now we proceed through the expectations and
verify that the invocation count satisfies the expectation. Later,
when we implement sequences, we can also verify that the order of
invocations is correct.\footnote{Aha! We can simply attach an
  invocation sequence number to each expectation as we invock a
  matching function. This will allow us to verify that the
  expectations in each sequence have a monotonically increasing
  invocation sequence number.}

We should be sure to proceed through the entire list of expectations,
and not necessarily short-circuit at the first matching
expectation. If that first matching expectation has already been
satisfied, then we will need to proceed to the next potentially
matching expectation. This means that we will need to perform
verification at each invocation step, so that we can easily skip
already-satisifed expectations.

\section{Automatically generating mock modules}
\label{application:generation}

Mocks are only truly useful as a testing tool when they can be
automatically generated from an interface. We will use Camlp4, an
OCaml preprocessing tool, a new OCaml feature called \textit{extension
  points} used for AST annotation, and the OCaml compiler's
\textit{compiler-libs} in order to generate and pretty-print OCaml
code.
