\documentclass[dissertation]{softeng}

\usepackage{times}
%% \usepackage[normalem]{ulem}
%% \usepackage{outline}
%% \usepackage{mdwlist} % should use enumitem anyway
\usepackage{listings}
\usepackage{hyperref}
\usepackage{graphicx}
\usepackage{xcolor}
\usepackage{emptypage}
\usepackage[right = 2in]{geometry}
\usepackage{marginnote}
\usepackage[nounderscore]{syntax}

% http://www.johndcook.com/blog/2008/11/24/how-to-put-pdf-properties-in-a-latex-file/
\hypersetup
{
    pdfauthor={Mike McClurg},
    pdfsubject={The implementation and usage of the Mock Object pattern in OCaml},
    pdftitle={MoCaml: the Mock Test Patter for OCaml},
    pdfkeywords={Software testing, functional programming, mock modules, OCaml, test generation},
}

%% Code formatting

\lstset{
  basicstyle=\ttfamily\footnotesize,
  breaklines,
  breakatwhitespace,
  keywordstyle=\color{blue},
  stringstyle=\color{mymauve},
  numberstyle=\tiny\color{mygray},
  title=\lstname,
  showstringspaces=false,
  commentstyle=\color{mygreen},
}
\definecolor{mygreen}{rgb}{0,0.6,0}
\definecolor{mygray}{rgb}{0.5,0.5,0.5}
\definecolor{mymauve}{rgb}{0.58,0,0.82}
\newcommand{\code}{\lstinline}

%% For writing margin notes while editing
\newcommand{\note}[1]{\marginnote{\small{{\color{red} #1}}}}

%% XXX title needs work
%\title{MoCaml: automatic dependency generation for OCaml}
\title{MoCaml: the Mock Test Pattern for OCaml}
\author{Mike McClurg}
\organisation{University of Oxford}
\college{Kellogg College}
\award{Software Engineering}

\date{\today}

\begin{document}

\maketitle

\begin{abstract}

  Object mocking is a technique to assist programmers in writing unit
  tests, by replacing hard-to-test dependencies with ``mock''
  implementations of those dependencies. A mock object allows the
  developer to set certain expectations about how the system under
  test will interact with the dependency, and then verify that these
  expectations hold true at the end of the test. While this technique
  was developed as a test design pattern for object oriented
  languages, it can also be useful for functional programming
  languages, especially for applications in which hard-to-test side
  effects can't easily be factored out of impure functions. This
  dissertation will explore the current state of the art tools and
  techniques for unit testing and mock object generation in object
  oriented languages. It will make the case that these techniques can
  be of use in functional programming languages, and it will describe
  the implementation and use of a new mocking library for the OCaml
  programming language.

  % ... Alas, functional languages are not immune to ...

\end{abstract}

\newpage

%% Dedication

\topskip0pt
\vspace*{\fill}
\begin{centering}
  \textit{This thesis is dedicated to my wife, without whom I likely
    would never have started writing.}
\end{centering}
\vspace*{\fill}

\newpage

\tableofcontents

\newpage

\listoffigures
%\listoftables

%% Include chapters
\chapter{Introduction}

\section{Remind me what this is about?}


\chapter{Background}
\label{background}

In this section we will provide a brief tour of the OCaml programming
language, focusing espectially on its powerful module system. The
chapter follows with an in-depth discussion of xUnit's Test Double
patterns and their implementation in OCaml and Java. The chapter
concludes with a summary of the current state-of-the-art testing tools
for functional languages such as OCaml and Haskell.

\section{A brief tour of OCaml and its module system}
\label{ocaml}


\section{Test Double patterns in detail}
\label{testdoubles}

% \section{The test double pattern, especially mocks}
\subsection{Dummy Object pattern}
\label{testdoubles:dummy}

\subsection{Fake Object pattern}
\label{testdoubles:fake}

\subsection{Test Stub pattern}
\label{testdoubles:stub}

\subsection{Test Spy Object pattern}
\label{testdoubles:spy}

\subsection{Mock Object pattern}
\label{testdoubles:mocks}

\section{Tools for software testing in functional langauges}
\label{testtools}

\begin{enumerate}
\item Haskell quickcheck
\item xUnit implementations: HUnit, OUnit
\item Kaputt for OCaml
\item Haskell benchmarking lib?
\end{enumerate}

\chapter{Application}

\section{I'd rather call ``Implementation,'' but I suppose it's more than that}


\chapter{Reflection}
\label{reflection}

\section{Comparison of mocking frameworks}
\label{application:comparison}

Compare MoCaml, manual mocking with Kaputt, and JMock. Demonstrate
pros and cons of each framework.

\subsection{Notes}

\begin{enumerate}
\item JMock can't tell at run-time whether an expecation
  type-checks. Make sure that we can do this!
\item \code{oneOf(mock).foo(); oneOf(mock).foo()} means \textbf{two}
  calls to \code{foo()}, not just one. (Note: this is okay because
  they occur in sequence. Two allowings with this behaviour would be a
  problem.)
\item JMock doesn't allow one to pass in methods or closures to
  implement mock functions, but we can (and should!) allow this.
\end{enumerate}

%% Bibliography
%\nocite{*}
\bibliographystyle{plain}
\bibliography{bibliography}

\end{document}
