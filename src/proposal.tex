\documentclass[proposal]{softeng}

\usepackage{times}

\title{MSc Project Proposal}
\author{Mike McClurg}
\organisation{University of Oxford}
\college{Kellogg College}
\award{Software Engineering}

\date{\today}

\begin{document}

\maketitle

\begin{abstract}

  %% You should include a 100 to 200 word abstract.

  Object mocking is a technique to assist programmers in writing unit
  tests, by replacing hard-to-test dependencies with ``mock''
  implementations of those dependencies. A mock object allows the
  developer to set certain expectations about how the system under test
  will interact with the dependency, and then verify that these
  expectations hold true at the end of the test. While this technique
  was developed as a test design pattern for object oriented
  languages, it can also useful for functional programming languages,
  especially for applications in which hard-to-test side effects can't
  easily be factored out of pure functions. This dissertation will
  explore the current state of the art tools and techniques for unit
  testing and mock object generation in object oriented languages. It
  will make the case that these techniques can be of use in functional
  programming languages, and it will describe the implementation and
  use of a new mocking library for the OCaml programming language.

  % ... Alas, functional languages are not immune to ...

\end{abstract}

\section{Area of study}

%% You should provide a brief account of the application domain: the
%% situation that you are going to explore, the software that you are
%% going to develop, or the problem that you wish to address.  This
%% section should be one or two pages long.

Testing is an important part of software development. Software tests
are often categorised as either system, component, or unit
tests. These categories represent a spectrum of granularity. Systems
tests represent the broadest granularity, and are meant to test the
entire system as a whole. On the other end of the spectrum are unit
tests, which test functionality at the finest level of granularity:
typically a method or function. Component tests sit somewhere in
between the two ends of the spectrum, and might test interactions
between modules or classes of an application, or perhaps the
interaction between a web server and a database. A system test may
require a test environment that is an exact replica of the production
system, whereas a unit test may be run on the developer's workstation
each time the software is rebuilt.

%% Systems tests test the entire system as a whole; for instance, a
%% system test for a web application may duplicate an entire production
%% environment to test the interaction of system components from end to
%% end. (System tests are meant to test the entire system as a whole; for
%% instance, a system test of a server virtualization system might
%% install the virtualization system on two physical hosts, install a
%% virtual machine on one of the hosts, and then assert that this virtual
%% machine can successfully migrate to the other physical host.)
%% Component tests are meant to test the interactions between individual
%% components of the system. These are distinct from systems test, which
%% test all the components of the system

Recent trends in Agile development, such as Continuous Integration
\cite{humble:continuous} and Test Driven Development \cite{beck:tdd},
have placed a greater emphasis on unit testing. Consequentially, many
tools and libraries have been developed to assist programmers in
writing these tests. Many of these libraries, such as JUnit
\cite{www:junit} for Java, NUnit \cite{www:nunit} for .NET, and OUnit
\cite{www:ounit} for OCaml, follow the xUnit patterns of test
development described by Meszaros \cite{meszaros:xunit}. This family
of libraries typically provides a set of convenient assertion
functions to help the user write test cases, annotations to help
organise these test cases into larger test suites, and test harnesses
to handle test running and status reporting.

A common problem with writing unit tests is that, for various reasons,
some code is just hard to test. This is often the case when the
depended-on component (DOC) of the system under test (SUT) performs
some side-effecting operation that is not valid in the test scenario,
or is otherwise impossible to run from within the test
harness. Hard-to-test code requires one to refactor the SUT so that
the hard-to-test DOC can be replaced by a test double
\cite{meszaros:xunit}. This test double might be a dummy object (such
as Java's \verb|null| object, which fulfils the DOC's interface but
doesn't implement it), a test spy (which implements the DOC's
interface and simply records each method access for later analysis),
or a mock object. Mocks are a powerful type of test double which allow
implementations to be easily specified in the test case; they also
allow ``expectations'' about the way the SUT interacts with the DOC to
be expressed and later verified. Mocks are a powerful tool which allow
one to write concise, maintainable unit tests that are less prone to
``brittle test syndrome'' than tests written differently
\cite{meszaros:xunit}. JMock \cite{www:jmock} is an excellent example
of a mocking library that automatically generates mock objects and
provides a very convenient method for specifying and evaluating
expectations on those mocks.

% Now I should transition to talking about unit testing in functional
% programming languages

While most of the literature about unit testing and tools for unit
testing are written about testing object oriented languages, many of
these techniques also apply directly to functional programming
languages. HUnit \cite{www:hunit} and OUnit \cite{www:ounit} provide
unit testing frameworks for Haskell and OCaml, respectively. Kaputt
\cite{www:kaputt} is another unit testing framework for OCaml which
provides a variety of modules to assist in writing tests, as well as
prototypical mocking combinators to assist in manually writing mock
functions. Unfortunately, Kaputt's mocking functions require the user
to manually write out mocks, and Kaputt doesn't provide the powerful
expectation specification mechanisms that other mocking libraries like
JMock \cite{www:jmock} provide.
% (Somewhere I need to talk about why Kaputt's mocking is not sufficiently powerful.)

% Talk about quickcheck, factoring pure from side-effecting code.
In fact, functional programming languages often lend themselves to
additional testing techniques that aren't as easily applied to object
oriented languages. Haskell and, to a lesser extent, OCaml, both have
access to a tool called QuickCheck \cite{canou:ocaml_random_test}
\cite{claessen:quickcheck}, which test the properties of functions by
generating random test inputs based on the type of the function under
test. Haskell unit testing in particular benefits from Haskell's
purity, or freedom from side effects. Because all side effects in
Haskell must be encapsulated in monads, it is easy to identify pure,
side effect free functions for which tests that don't require test
doubles are easy to write. Most object oriented languages don't
provide this level of surety about the lack of side effects in a
function.

%% Talk about how the above techniques don't always apply simply
%% because of the nature of the application domain or the way in which
%% a legacy application was written, and bring us back to why and how
%% mocks could be useful in a functional programming language such as
%% OCaml.

Most functional languages don't enforce the level of purity that
Haskell does. OCaml, for instance, does allow for mutable state, and
doesn't force the programmer to separate side effecting code from
referentially transparent code. So while OCaml encourages good
programming practices such as separating referentially transparent
code from side effecting code, and minimising mutable state, it
doesn't enforce these practices. Furthermore, some application domains
are by their nature more prone to requiring a large portion of the
code base to produce or deal with side effects. In situations like
these, the ability to easily swap out hard-to-test dependencies with
mechanically generated test doubles is very important to the
developer who wants to write good test cases quickly and easily.

\section{Proposed work}

%% You should explain what you intend to do, how you propose to do it,
%% and what you expect to achieve in terms of outcomes.  It should be
%% clear from your explanation:
%% \begin{itemize}
%% \item which aspects of the situation, system, or problem that you intend
%%   to address---the intended scope of your project;
%% \item which principles, methods, tools, or techniques you intend to
%%   apply;
%% \item what you expect to be able to report, and how this will serve to
%%   demonstrate your mastery of the subject.
%% \end{itemize}
%% This section also should be one or two pages long.

The main output of my project will be MoCaml, a mocking library for
OCaml which mechanically generates mocks and provides an expecataion
framework which will make it easy for developers to write test cases
using these mocks. This library should integrate well with existing
OCaml unit testing libraries, and will likely build upon them.

As mentioned previously, the Kaputt unit testing library does have a
set of helper functions for manually creating mocks, and for
performing limited expectation checking on these mocks. This library
is a step in the right direction for OCaml, but it is incomplete when
compared against libraries such as Java's JMock, which have automatic
mock generation capabilities, as well as a convenient and powerful
syntax for specifying expectations. Kaputt's existing mocking helper
functions may be useful as a starting point for automatically
generating mocks in MoCaml, though the Kaputt mock library is not
required for implementing MoCaml.

The JMock library makes extensive use of Java's reflection
capabilities to generate mock objects at runtime. While the OCaml
language lacks reflection, it does have a very powerful preprossessor
and macro facility called camlp4 \cite{www:reading_camlp4}. As a
preprocessor, camlp4 is much more akin to Lisp's syntax macros than to
C's preprocessor directives, providing the ability to write syntax
extensions as well as direct access to a program's untyped AST. Camlp4
will be powerful enough to do compile-time mock generation, as well as
allow us to write just enough extra syntax to make writing and using
mock and expectations convenient.



\section{Project plan}

%% You should explain the order in which you will address the various
%% aspects of the work, a realistic allocation of time to each task, and
%% the dates by which each task should be completed.  You should list key
%% milestones and interim outputs or results, and explain what actions
%% would be taken if some of these could not be achieved: what would you
%% do instead?   This section should be between half a page and perhaps a
%% whole page in length.

\section{Dissertation structure}

%% You should describe the broad organisation of your dissertation, an
%% outline table of contents, including chapter headings and subheadings
%% would be fine.

\nocite{*}
\bibliographystyle{plain}
\bibliography{bibliography}

\end{document}
