\documentclass[proposal]{softeng}

\usepackage{times}

\title{MSc Project Proposal}
\author{Mike McClurg}
\organisation{University of Oxford}
\college{Kellogg College}
\award{Software Engineering} 

\date{\today}

\begin{document}

\maketitle

\begin{abstract}
You should include a 100 to 200 word abstract.
\end{abstract}
 
\section{Area of study}

You should provide a brief account of the application domain: the
situation that you are going to explore, the software that you are
going to develop, or the problem that you wish to address.  This
section should be one or two pages long. 

\section{Proposed work}

You should explain what you intend to do, how you propose to do it,
and what you expect to achieve in terms of outcomes.  It should be
clear from your explanation:
\begin{itemize}
\item which aspects of the situation, system, or problem that you intend
  to address---the intended scope of your project;
\item which principles, methods, tools, or techniques you intend to
  apply;
\item what you expect to be able to report, and how this will serve to
  demonstrate your mastery of the subject.
\end{itemize}
This section also should be one or two pages long. 

\section{Project plan}

You should explain the order in which you will address the various
aspects of the work, a realistic allocation of time to each task, and
the dates by which each task should be completed.  You should list key
milestones and interim outputs or results, and explain what actions
would be taken if some of these could not be achieved: what would you
do instead?   This section should be between half a page and perhaps a
whole page in length.  

\section{Dissertation structure}

You should describe the broad organisation of your dissertation, an
outline table of contents, including chapter headings and subheadings
would be fine.

\nocite{*}
\bibliographystyle{plain}
\bibliography{bibliography}

\end{document}
