\chapter{Reflection}
\label{reflection}

\section{Reflection on our work}

%% \begin{enumerate}
%% \item Revisit unit testing and the purpose of mocking (to provide both
%%   indirect inputs and outputs to a test case, and to verify
%%   expectations of behavior placed on the mocked module by the test
%%   writer).
%% \item Describe the results of our work, the MoCaml library (or
%%   ppx\_mock?). What have we accomplished? Can this be installed?
%% \end{enumerate}

In this thesis we have discussed unit testing theory, and described
it's application to both object oriented and functional programming
languages. Specifically, we have investigated the ``mock object''
pattern, and described how it could be used in the strongly typed,
functional programming language OCaml. We then showed how the mock
patter could be implemented in OCaml using its preprocessing and code
generation facilities.

\section{Comparison of mocking frameworks}
\label{application:comparison}

%% Compare MoCaml to: manual mocking with Kaputt, and JMock. Demonstrate
%% pros and cons of each framework.

%% \begin{enumerate}
%% \item JMock can't tell at run-time whether an expectation
%%   type-checks. Make sure that we can do this! (We can :) )
%% \item JMock doesn't allow one to pass in methods or closures to
%%   implement mock functions, but we can (and should!) allow this.
%% \item MoCaml and Kaputt both support count (and order? behavior?)
%%   expectation types. Kaputt requires more work than we would like in
%%   order to set up a mock.
%% \end{enumerate}

As shown in Chapter \ref{background}, mocking frameworks exist for
many object oriented languages, as well as for the OCaml
language. MoCaml compares very favourably to the Kaputt testing
framework for OCaml (which provides basic functionality for manually
implementing mock functions), and it is approaching ``core-feature''
parity with the likes of JMock, the quintessential mocking framework
for Java, and the reference implementation of the original Mock Object
Pattern.

% How is it better than Kaputt? Automatic generation.
\subsection{Comparison to the Kaputt library for OCaml}

The Kaputt unit testing library for OCaml provides simplistic mock
building facilities, which assist the test writer in preparing mock
functions. Mock functions written with the Kaputt library have the
ability to record function invocations, and to transparently invoke
the original function. Compared to the MoCaml library implementation
described in Chapter \ref{application}, the Kaputt library essentially
provides the ability to write ``function proxies,'' as described in
section \ref{application:mock_verification}. Other limitations of the
Kaputt library's mocking facilities include:

\begin{itemize}
\item No automatic generation of mock functions.
\item Provides no assistance with dependency-injection of the new mock
  functions.
\item Built-in mocking combinators only support functions with up to
  five arguments, and no optional argument support.
\item In order to mock functions with more than one argument, one must
  first manually uncurry the function to be mocked.
\item Validation of mocks is limited to assertions on inputs, and
  invocation counts; no order validation is provided.
\item No expectation language for specifying expectations.
\end{itemize}

Using the Kaputt library for writing tests with the mock pattern is
tedious because of the lack of code generation for mock modules and
functions. The lack of an expectation language for describing
expectations makes it difficult to describe complex expectations, and
makes it difficult for future programmers to maintain the test
cases. MoCaml makes up for all of these limitations with it's
expectation language and automatic mock module generation
capabilities.

\subsection{Comparison to the JMock library for Java}

%% How is it similar to JMock? Auto-gen, some expectations, dsl for
%% expectations.

The JMock library \cite{freeman:evolving} \cite{freeman:growing}
\cite{www:jmock} is one of the major mocking frameworks for the Java
language, and was the inspiration for the MoCaml library, and MoCaml
strives to reach feature parity with JMock. Like JMock, the MoCaml
library provides mocking facilities for a strongly typed language; it
can generate mock module implementations from a module type
specification; there is a light-weight expectation language that can
be used to define the expectations placed on a mock; and MoCaml
supports many of the same expectations that JMock supports.

%% How is it lacking compared to JMock? Not as many expectation types,
%% fewer options for validation, DSL is ``less fluent'' (maybe?),
%% requires a module type signature (equivalent of a Java interface)

JMock is, however, a much more mature piece of software, and thus has
features that MoCaml lacks. For instance, JMock supports more
invocation counting expectations than MoCaml does, such as specifying
a range of invocations which a method will be called, or allowing a
method to be called without requiring it. JMock makes use of Java's
reflection capabilities to generate mock implementations at run time,
while MoCaml requires a separate compilation step in order to
statically generate the mock implementation. Because of the
limitations of the OCaml preprocessor system, MoCaml can only generate
mock module implementations from a module type signature (roughly
equivalent to a Java interface), and not from a module's
implementation, which doesn't have enough type information for the
preprocessor to operate with. JMock also provides a richer exception
language, with the capability to define expectations on method
parameters with logical operators, and even to extend the expectation
language with new actions. Some of these shortcomings may be resolved
with more development effort put into MoCaml, while others, such as
requiring a module type signature and a two-stage compilation
approach, are limitations imposed by the way in which MoCaml is
implemented.

\section{Next steps for MoCaml and mock modules for OCaml}

%% \textit{Maybe call this ``Next steps''}

We have covered a lot of ground in this dissertation, but there is
still more work that can be done, both to the MoCaml library and for
general work on software testing in functional languages. It is yet to
be seen whether mocking will become a widely used practice in the
OCaml community, as neither MoCaml, nor mocking, has (to the author's
knowledge) been used for unit testing in any large-scale OCaml
software.

There is no reason that the mock technique needs to be relegated
solely to the object oriented language community -- truly, the
technique has much less to do with objects than it does with replacing
side-effecting code. The greatest benefit of mocking is that it allows
one to write tests for hard-to-test code; often the reason for code
being hard-to-test is that it produces side effects. The mocking
technique is excellent for replacing networking, file IO, and database
manipulation code, all of which are common to both functional and
object oriented programming. It would seem that the mocking technique
would be a welcome addition to a functional programmer's toolkit,
especially for a language such as OCaml, which encourages, but doesn't
enforce, the separation of side-effecting from non-side-effecting
code.

Before the MoCaml library itself will be useful for large-scale
commercial OCaml code bases, some more work must be done. Most
importantly, and most unfortunately, the code generation portion of
the MoCaml library is still incomplete. Further to this deficit, there
are a number of issues which the author would like to remedy in the
current implementation. Namely:

\begin{description}

\item [A module's values may be shadowed by generated values]

  We implement the mock module by generating a large amount of new
  code, including functions and types, as well as inserting static
  helper functions. If there is a type or value in the original module
  which shares the same name as a generated name, the MoCaml generated
  name will shadow the original one (this is a common problem for macro
  writers). We can prefix these new types and values with some string
  tag which serves to differentiate MoCaml-generated identifiers from
  user-defined values, but this is not a total solution. It would be
  preferable if the implementation did not have the potential for
  unintentionally shadowing values from the original module.

  %Shadowing of mocked module's values with injected values for
  %operating the mock. We can fix this, right?
  
\item [More expectations]

  We would like to provide invocation counting constructs such as ``at
  least $n$,'' ``at most $n$,'' and ``between $n$ and $m$''. We would
  like to be able to specify expectations on function parameters
  directly, instead of requiring the test-writer to supply a function
  which does so.

  %% maybe some more actions, or more complex ordering?

\item [More control over expectation validation]

  The current implementation of MoCaml performs some expectation
  validations inline with test execution, and performs others after
  the test has completed. We would like to provide more control over
  this behaviour, perhaps allowing for a list of violations to be
  presented to the test-writer at the end of the test execution. Also,
  it could be useful to provide a strictness option over invocation
  counting. This option would allow test writers to specify whether a
  set of expectations define exactly the set of invocations required,
  or provide only a minimal set of required invocations.

\item [Thread safety]

  We use mutable state to keep track of invocations, and in the future
  we may also keep track of violations, instead of raising exceptions
  as violations are encountered (see above). Because of this mutable
  state, MoCaml's implementation is not currently thread safe. It may
  be significant work to properly verify ordering in the face of
  threaded execution, so modifying the library to support threaded
  execution would take much thought.

\item [Nicer syntax for specifying expectations]

  A major issue that must be overcome when generating type-safe mock
  modules is allowing the user to specify expectations in a type-safe
  manner, while still providing simple action expectation syntax. The
  current implementation relies on generated polymorphic variants
  which carry function implementations and return values for mocked
  functions. The resulting syntax for specifying action expectations
  is a bit redundant. We would like to investigate the use of
  Generalised Algebraic Data Types (GADTs) to provide an approximation
  of dependent types, which could allow for a more elegant expectation
  language. This solution may be similar to that of the recent work to
  replace OCaml's string formatting library with a GADT-based
  implementation \cite{vaugon:gadt-format}. A potentially simpler
  method which works outside of the type system may be to use OCaml's
  preprocessing facilities to rewrite the more elegant expectation
  language which we desire into the uglier, ``real'' implementation.

%% \item [Publish the MoCaml library]

%%   A well written dissertation is a fine achievement, but it doesn't do
%%   much to help the professional OCaml programmer write better
%%   tests. Finishing the MoCaml library and publishing it will do more
%%   to futher the practice of good software construction than this
%%   thesis will on its own.

\end{description}

The mock pattern and mocking libraries have done much to further the
state-of-the-art in software testing for object oriented
languages. The mock pattern's ability to remove side effects makes it
easy to test what once was hard-to-test code. While it may be easier
to develop a mocking library in languages with run-time reflection
capabilities, or even dynamically typed languages, it is entirely
possible to construct a mock-generation framework using just the
facilities that are provided by the OCaml compiler itself. The mock
pattern has the potential to help OCaml programmers write more tests,
and to write them faster than they could before, and that in turn has
the potential to greatly increase software quality as a whole.
