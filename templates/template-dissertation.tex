\documentclass[dissertation]{softeng}

\usepackage{times}

\title{Producing your MSc dissertation \\ using \LaTeX}
\author{Jim Davies}
\college{Kellogg College}
\organisation{University of Oxford}
\award{Software and Systems Security}

\begin{document}

\maketitle

\begin{abstract}
  The abstract should really really really be between 100 and 200
  words in length.
\end{abstract}

\clearpage

\begin{acknowledgements}
  Thanks.  
\end{acknowledgements}

\clearpage

\pagenumbering{roman}
\pagestyle{plain}
\setcounter{tocdepth}{2}

\tableofcontents

\clearpage

\pagenumbering{arabic}
\pagestyle{myheadings}

\chapter{Introduction}

This document is not a sample dissertation, but a simple template
created to illustrate the use of the \verb|softeng| class with
\LaTeX.  Although the bulk of the information is presented as an
appendix, you might like to know now that: 
\begin{itemize}
\item the \verb|softeng| class is configured for producing a proposal
  document by default, with sections rather than chapters.  To use it
  in `dissertation mode', you need to supply the \verb|dissertation|
  argument, as we have done here at the top of the file.
\item if you have any problems with the `oxfordlogo' graphic, or if
  you just prefer a plainer coverpage, you can add the \verb|nologo|
  option, which has the expected effect. 
\end{itemize}
For everything else, take a look at this
file---\verb|dissertation.tex|---and it may become clear. 

In particular, copy the commands that define \verb|title|,
\verb|author|, \verb|college|, \verb|organisation|, and \verb|award|
so that they apply to you.  The last of these should be set to either
`Software Engineering' or `Software and Systems Security', obviously.

Please do write and tell us if you think that there is something that
we could have said here that would have saved you time in using
{\LaTeX} for your dissertation.  This is the first version of the
class file, and the first version of this document.  We aim to improve
it, but we'll be able to do that more quickly with your help.

\chapter{Background}

You'll probably want to choose a more specific title for this
chapter.  It should be between two and ten pages in length (the
chapter, not the title).  

\chapter{Application and reflection}

In this part of the dissertation you will report and reflect upon your
project work~\cite{exampleref}.  You might organise this as two, or more,
separate chapters, representing 40 to 60 pages of material in all.

\chapter{Conclusion}

That's it.  And here comes the bibliography. 

\bibliographystyle{plain}
\bibliography{template-bibliography}

\appendix

\chapter{Notes on \LaTeX}
\label{chap:latex}

There are books that you can buy on \LaTeX, but you are unlikely to
need any of them: the commands mentioned here should see you through.
If there's something else that you think we should add, let us know!

\section{Letters, words, lines, and paragraphs}

Remember, don't break your chapters up like this unless you really are
presenting something along the lines of a manual.   

\subsection{Special characters}

{\LaTeX} treats the following characters as special cases: 
\begin{itemize}\raggedright
\item \verb|\| is the escape character, and indicates the beginning of
  a command, as in \verb|\chapter|; if you want a backslash symbol,
  write \verb|\backslash| \emph{in math mode (see below!)}
\item \verb|%| comments out the remainder of the line, so if you need
  a percent symbol, write \verb|\%|
\item \verb|$| puts you into math mode, which you probably won't need
  unless you're typesetting some mathematics; if you want a dollar
  symbol, write \verb|\$|.  If you want the pound sign, type
  \verb|\pounds|. 
\item \verb|{| opens a group of characters for interpretation, and
    \verb|}| closes one.  This can be used to delimit an argument to a
  command: e.g. \verb|\chapter{Name}|.  It can also be used to limit
  the scope of some declaration: e.g. 
  \verb|{\bfseries some text in bold}|.  
  If you want the braces themselves, you'll need \verb|\{|
  and \verb|\}|.
\item \verb|&| is used as a column separator in the \verb|\tabular|
  environment: see Section~\ref{sec:tabular} below.  If you need an
  ampersand symbol, you can write \verb|\&|. 
\item \verb|[| and \verb|]| after a command are used to delimit an
  optional argument.  If you find that this is happening when you
  don't want it to, then you can put \verb|\relax| between the command
  and the opening square bracket---e.g. 
  \verb|\item \relax [some text in brackets]|
\item \verb|`| and \verb|'| produce opening and closing quotation
  marks, respectively; they do the right thing when doubled.  Take
  care to use these correctly!
\end{itemize}

\subsection{Breaking}

{\LaTeX} does this kind of thing for you, but you need to show it how
your text is divided into paragraphs.  Just leave a blank line to do
this.  If you can't leave a blank line (really?) you can type
\verb|\par| instead.

In running text, a new paragraph should start with an indented line,
unless it's the first paragraph after a heading.  This will happen
automatically.  Don't mess with this: that is, don't redefine
\verb|\parindent| and \verb|\parskip|.  If you need to tell a specific
paragraph \emph{not} to indent the first line, and you really mean to
do that, then you can type \verb|\noindent| just before it starts.  

\begin{itemize}\raggedright
\item \verb|\\| produces an immediate line break.  Do not use this in
  running text unless you absolutely have to; never use it to create
  paragraph breaks---just use a blank line. 
\item Use \verb|\clearpage| to produce a page break.  You shouldn't
  really do this kind of thing until you're putting the finishing
  touches to your document.
\end{itemize}

If {\LaTeX} is complaining about ``overfull hboxs'', that's probably
because it can't break a line where it wants to.  If the box in
question is overfull by more than 10pt (take a look at the typeset
version and see if you can spot it) then try reorganising the line or
paragraph.

\subsection{Fonts}

You might not need to insert any font commands in your running text,
but in case you want to: 
\begin{itemize}\raggedright
\item \verb|\emph| puts its argument in emphatic font---usually italic
  or slanted: e.g.\\ ``\verb|\emph{your} mileage|'' produces
  ``\emph{your} mileage''.
\item
  \verb|\verb| does more than just put its argument in typewriter 
  font, it renders it verbatim: spaces matter. %
  Exceptionally, this command requires that its argument be delimited
  by the next symbol it sees: %
  e.g.  \texttt{$\mathtt\backslash$verb|st\ uff|} will produce
  \verb|st uff|.  Also, the argument to
  \verb|\verb| can't include a linebreak: for that, you need a
  \verb|verbatim| environment.
\item \verb|\textbf|, \verb|\textsl|, \verb|\textit|, \verb|\textsf|,
  and \verb|\texttt| put their arguments in bold, slanted, italic,
  sansserif, or typewriter font, respectively.
\item \verb|\bfseries|, \verb|\slshape|, \verb|\itfamily|,
  \verb|\sffamily|, and \verb|\ttfamily| are the corresponding
  declarations, and apply for the rest of the current enclosing group.
\item \verb|\small| and \verb|\large| do what their names suggest: you
  might want to use \verb|\small| to shrink a table to fit, but
  otherwise you shouldn't need to use these.  
\end{itemize}

\section{Environments}

\subsection{Lists}

Environment commands in {\LaTeX} are more like the familiar markup of
HTML or XML: they have a \verb|\begin| and an \verb|\end|.  
\begin{itemize}\raggedright
\item \verb|\begin{enumerate} \item ... \item ... \end{enumerate}|
  produces an enumerated list, with command \verb|\item| marking the
  beginning of a new item with a label. 
\item \verb|\begin{itemize} \item ... \item ... \end{itemize}| does
  the same, but the labels are all bullets, or dashes, or circles,
  depending upon how deep you go.
\item \verb|\|\verb|begin{verbatim} ... \end{verbatim}| renders
everything inside it, well, in verbatim: in typewriter font, with
newlines and spaces, and even the special symbols, appearing as they
are.  This is ideal for presenting program code or model text.
\item \verb|\begin{quote} ... \end{quote}| indents the enclosed
  material to the same extent as an itemized or enumerated list.
\end{itemize}
The {\LaTeX} \verb|description| environment is also available, but is
a little awkward in comparison to those above.  A better solution may
be to use a \verb|\subsubsection| or \verb|\paragraph| command to
introduce each of the items, if you absolutely have to do this.  

\subsection{Figures and tables}
\label{sec:tabular}

Figures and tables are \emph{floats} in \LaTeX: they will appear near
to the surrounding text if they can, but they might also float away
only to wash up later in the document.  
\begin{itemize}\raggedright
\item A figure environment can be used to import a graphic into the
  document (don't use the archaic {\LaTeX} drawing commands unless you
  have a good reason to).  Any graphic will do, and you shouldn't
  need to specify an extension.  
\begin{verbatim}
  \begin{figure}[t]
    \centering
    \includegraphics{layercake}

    \caption{How to have your cake and eat it}
    \label{fig:layercake}
  \end{figure}
\end{verbatim}
  The \verb|t| option says `put this at the top of a page'.  \verb|h|
  says `put it here instead'.  You can add an exclamation mark in each
  case to (try to) insist: e.g. \verb|t!|.  You shouldn't have to,
  though. 

  The \verb|\caption| command creates a caption using the supplied
  text, and the \verb|\label| command defines a label for referring to
  this figure in the text (see Section~\ref{sec:labelling} below).
\item The \verb|\includegraphics| command accepts optional arguments
  in square brackets before the name of the graphic file.  These can
  be used to determine the size and orientation of the picture.  The
  most useful are:
  \begin{itemize}
  \item \verb|\includegraphics[scale=0.8]{...}| to scale the graphic
    by, for example, \verb|0.8|, or 80\%.
  \item \verb|\includegraphics[width=6cm]{...}| to set the
    width of the graphic to, for example, 6cm.  
  \item \verb|\includegraphics[height=6cm]{...}| to set the
    height of the graphic to, for example, 6cm.  
  \end{itemize}
\item A table environment is pretty much the same thing, although
  you're quite likely to be using {\LaTeX}'s table making commands
  instead of importing a table drawn in another package and exported
  as a graphic.  
\begin{verbatim}
  \begin{table}[t]
    \centering
    \begin{tabular}{l|l}
      Person & Score \\ \hline
      Susan & 30 \\
      Peter & 40 
    \end{tabular}
    \caption{Tennis score when rain stopped play}
    \label{tab:tennis}
  \end{table}
\end{verbatim}
\item You can also set tables as displayed material in running text:
  in this case, you want to use the \verb|tabular| environment,
  perhaps inserted within a \verb|quote| environment or a
  \verb|center| environment (note the spelling) to get the alignment
  correct.
\begin{verbatim}
  \begin{quote}
    \begin{tabular}{l|l}
      Person & Score \\ \hline
      Susan & 30 \\
      Peter & 40 
    \end{tabular}
    \caption{Tennis score when rain stopped play}
    \label{tab:tennis}
  \end{quote}
\end{verbatim}
  produces: 
  \begin{quote}
    \begin{tabular}{l|l}
      Person & Score \\ \hline
      Susan & 30 \\
      Peter & 40 
    \end{tabular}
  \end{quote}
  Note that there is no caption here, nor is there a label. 
\item The \verb|tabular| command takes a `table specification' as a
  mandatory argument:
  \begin{itemize}
  \item \verb|l|, \verb|c|, and \verb|r| declare columns with material
    to be aligned to the left, centred, and aligned to the right,
    respectively.
  \item \verb+|+ denotes a vertical line.
  \end{itemize}
  After this argument comes the beginning of the first row. 

  Each row can contain up to $n - 1$ ampersands (`\verb|&|'), where
  $n$ is the number of columns declared: these act as delimiters
  between column material.  A row is ended by a double backslash
  (`\verb|\\|'), which takes an optional argument indicating any
  additional vertical space required before the next row:
  e.g. \verb|\\[2cm]|.  A \verb|\hline| after \verb|\\| produces a
  horizontal rule across all of the columns before the next row.  You
  can use two adjacent \verb|\hline| commands to produce a double
  rule.
\end{itemize}

\section{Sections, labels, references, and citations}

\subsection{Sections}

Your dissertation should be divided into a number of chapters, each
introduced using a \verb|\chapter| command with a single, mandatory
argument: the name of the chapter.  For example, this chapter, in the
appendix of the template document, begins with the command
\begin{verbatim}
  \chapter{Notes on \LaTeX}
\end{verbatim}
Your chapters can be divided into sections, and into subsections,
using the \verb|\section| and \verb|subsection| commands,
respectively.  Again, these take the name of the section or subsection
as their single, mandatory argument.  

Further levels of sectioning are available---\verb|\subsubsection|,
\verb|paragraph|, and \verb|subparagraph|---but try not to use these
unless what you are writing is intended to resemble a reference
manual.   

\subsection{Labels and (cross-)references}
\label{sec:labelling}

You should use labels to identify any chapters, sections, subsections,
items (in enumerated lists), figures, or tables that you wish to refer
to.  You can do this using the \verb|\label| command, with whatever
text you wish to use as a key: for example, 
\begin{verbatim}
  \chapter{Notes on \LaTeX}
  \label{chap:latex}
\end{verbatim}
might be an appropriate label for this chapter.  You can then refer to
this chapter by writing 
\begin{verbatim}
  Chapter~\ref{chap:latex}
\end{verbatim}
and {\LaTeX} will insert the appropriate number (provided that you've
run it at least once since inserting the corresponding \verb|label|).
Since we're in the first chapter of the Appendix here, this is going
to produce 
\begin{quote}
  Chapter~\ref{chap:latex}
\end{quote}
For floats---figures and tables---you should insert the \verb|\label|
command after the \verb|\caption|.  Note that since you are referring
to a specific item, you should capitalise `Chapter', `Section',
`Figure', or `Table'.  It is sufficient to say `Section' when
referring to smaller divisions: e.g. subsections. 

\subsection{Citations}

Although you can construct and maintain a bibliography by hand, it is
far more efficient to make use of the \verb|BibTeX| program.  To do
this, you create a \verb|.bib| file to hold the references that you
might wish to use.  You then place the commands
\begin{verbatim}
  \bibliographystyle{plain}
  \bibliography{example}
\end{verbatim}
at the appropriate point in your document, where \verb|example| is
the name of your \verb|.bib| file (the extension is not required). 

Each entry in your \verb|.bib| file will start with a key (after an
\verb|@| command identifying the document type).  For example,  
\begin{verbatim}
@article{exampleref,
  author = {Barry Andrews},
  title = {{Preparing an MSc Project Proposal}},
  journal = {What Project? Monthly},
  publisher = {Random House},
  year = {2009}
}
\end{verbatim}
The key here is `\verb|exampleref|'.  We can now `cite' this article
in our document, simply by using this key as an argument to the
\verb|\cite| command.   For example,
\begin{verbatim}
most of this information will be useful to those preparing a project
proposal~\cite{exampleref}
\end{verbatim}
will produce
\begin{quote}\raggedright
  most of this information will be useful to those preparing a project
  proposal~\cite{exampleref}
\end{quote}
To cite more than one reference at the same point, simply include the
respective keys in the argument to \verb|\cite|, separated by commas.

\section{Extras}

If you want to use mathematics, you should obtain the \verb|zed.sty|
package: email \url{Jim.Davies@comlab.ox.ac.uk} to ask.  If you want
to use fonts other than the default \emph{Computer Modern} (CM), then
you might like to try adding
\begin{verbatim}
  \usepackage{times}
\end{verbatim}
between then \verb|\documentclass| and \verb|\begin{document}|
  commands.  \verb|newcent|, \verb|utopia|, \verb|palatino| are other
  packages that define alternative fontsets.  It's probably best to
  stick with CM or Times, however.  

\end{document}

